\documentclass[11pt]{article}
\usepackage[utf8]{inputenc}
\usepackage[slovene]{babel}

\usepackage{amsthm}
\usepackage{amsmath, amssymb, amsfonts}

\theoremstyle{definition}
\newtheorem{definicija}{Definicija}[section]

\theoremstyle{theorem}
\newtheorem{uporaba}{Uporaba}[section]

\newtheorem{lema}{Lema}
\newtheorem{trditev}{Trditev}
\newtheorem{izrek}{Izrek}
\newtheorem*{dokaz}{Dokaz}
\newtheorem*{posledica}{Posledica}
\newtheorem*{dogovor}{Dogovor}

\title{Analiza 2 - definicije, trditve in izreki}
\author{Oskar Vavtar}
\date{2020/21}

\begin{document}
\maketitle
\pagebreak
\tableofcontents
\pagebreak

% #################################################################################################

\section{NEDOLOČENI INTEGRAL IN POJEM \\ DIFERENCIALNE ENAČBE}
\vspace{0.5cm}

% *************************************************************************************************

\subsection{Primitivna funkcija in nedoločeni integral}
\vspace{0.5cm}

\begin{definicija}[Primitivna funkcija]

Naj bo $f$ funkcija \textit{ene spremenljivke}. Če $\exists$ \textit{odvedljiva funkcija} $F: A \rightarrow \mathbb{R}$, za katero velja $F' = f$, imenujemo F \textit{primitivna funkcija} funkcije $f$ na $A$. 

\end{definicija}
\vspace{0.5cm}

\begin{definicija}[Nedoločeni integral]

\textit{Nedoločeni integral} funkcije $f$ je skupek vseh njenih primitivnih funkcij. Označimo ga z $\int f(x) dx$, funkcijo $f$ pa imenujemo \textit{integrand}.

\end{definicija}

\begin{posledica}

Naj bo $F$ neka \textit{primitivna} funkcija za $f$ na intervalu $J$. Potem je za $x \in J$
$$\int f(x) dx = F(x) + C,$$
kjer je $C \in \mathbb{R}$ poljubna konstanta, ki jo imenujemo \textit{splošna} ali \textit{integracijska konstanta}.

\end{posledica}
\vspace{0.5cm}

% *************************************************************************************************

\subsection{Uvedba nove spremenljivke v nedoločeni integral}
\vspace{0.5cm}

% *************************************************************************************************

\subsection{Integracija po delih v nedoločenem integralu}
\vspace{0.5cm}

% *************************************************************************************************

\subsection{Diferencialne enačbe 1.reda}
\vspace{0.5cm}

\begin{definicija}

\textit{Navadna diferencialna enačba} 1.reda je enačba za neznano funkcijo 
$$y=g(x),$$
ki vsebuje tudi odvod $y'$ funkcije $y$.

\textit{Splošna oblika} diferencialne enačbe 1.reda je
$$F(x,y,y')=0,$$
kjer je $F$ funkcija treh spremenljivk, ki je res odvisna od zadnje spremenljivke.

\textit{Splošna rešitev} diferencialne enačbe 1.reda je funkcija 
$$y=g(x,C)$$
(lahko podana implicitno),ki je odvisna od \textit{splošne konstante C} in reši dano diferencialno enačbo za poljubno izbiro vrednosti konstante $C \in \mathbb{R}$, poleg tega pa za poljuben začetni pogoj $\exists$ vrednost konstante $C$, pri kateri rešitev zadošča izbranemu začetnemu pogoju.

Rešitev, ki ne vsebuje splošnih konstant, imenujemo tudi \textit{posebna} ali \textit{partikularna rešitev}.

\end{definicija}

\begin{definicija}[LDE 1.reda]

\textit{Linearna diferencialna enačba} 1.reda ima obliko
$$r_1(x)y' + r_0(x)y = s(x),$$
kjer so $r_0, r_1, s: J \rightarrow \mathbb{R}$ funkcije, definirane na nekem intervalu J. Če je $s$ ničelna funkcija, rečemo, da je enačba \textit{homogena}. Če sta funkciji $r_0$, $r_1$ \textit{konstantni}, pa rečemo, da ima enačba \textit{konstante koeficiente}.

\textit{Standardna oblika} linearne diferencialne enačbe 1.reda je
$$y' + p(x)y = q(x),$$
kjer sta $p, q: J \rightarrow \mathbb{R}$ funkciji, definirani na intervalu $J$.

\end{definicija}
\vspace{0.5cm}

% *************************************************************************************************

\pagebreak

% #################################################################################################

\section{DOLOČENI INTEGRAL}
\vspace{0.5cm}

% *************************************************************************************************

\subsection{Motivacija za določeni integral}
\vspace{0.5cm}

\begin{definicija}

Naj bo $f:[a, b] \rightarrow \mathbb{R}$ \textit{nenegativna funkcija}, torej $f(x) \geq 0$ za vse $x \in [a, b]$. Rečemo, da graf funkicje f \textit{določa območje} $A \subset \mathbb{R}^2$  nad intervalom $[a, b]$. Množica $A$ je navzgor omejena z grafom funkcije $f$, na levi s premico $x=a$ in na desni s premico $x=b$.

\end{definicija}
\vspace{0.5cm}

% *************************************************************************************************

\subsection{Riemannova vsota in Riemannov integral}
\vspace{0.5cm}

\begin{definicija}[Riemannova vsota]

\textit{Delitev} $D$ intervala $[a, b]$ na podintervale je dana z izbiro \textit{delilnih točk} $x_i$:
$$a=x_0 < x_1 < x_2 < \ldots < x_{n-1} < x_n=b,$$
kjer je $n \in \mathbb{N}$. Dolžino $i$-tega podintervala $[x_{i-1},x_i]$ (za $i=1,2,...,n$) označimo z $\delta_i := x_i - x_{i-1}$. \textit{Velikost delitve} $D$ je dolžina najdaljšega podintervala delitve $D$, torej
$$\delta(D) = \max{\{\delta_i \mid i = 1, 2, ..., n\}}.$$

Na vsakem od podintervalov, na katere delitev $D$ razdeli interval $[a, b]$, izberemo \textit{testno točko} $t_i \in [x_{i-1}, x_i]$ in s $T_D = (t_1, t_2, \dots, t_n)$ označimo nabor teh točk; nabor testnih točk je \textit{usklajen} z delitvijo $D$, ker smo na vsakem podintervalu $[x_{i-1},x_i]$, določenem z $D$, izbrali natanko eno testno točko $t_i$.

\textit{Riemannova vsota} funkcije $f:[a, b] \rightarrow \mathbb{R}$, pridružena delitvi $D$ in usklajenemu naboru testnih točk $T_D$ je 
$$R(f, D, T_D) := \sum_{i=1}^{n} f(t_i) \delta_i.$$

\end{definicija}
\vspace{0.5cm}

\begin{definicija}[Riemannov integral]

\textit{Riemannov integral} ali \textit{določeni integral} funkcije $f:[a, b] \rightarrow \mathbb{R}$ je limita Riemannovih vsot $R(f, D, T_D)$, kjer limito vzamemo po $\forall$ delitvah D intervala [a, b] in usklajenih naborih testnih točk $T_D$, ko pošljemo velikost delitev $\delta(D)$ proti $0$, če ta limita $\exists$ (torej je končna in neodvisna od izbire delitev in testnih točk). Pišemo
$$\int_{a}^{b} f(x) dx := \lim_{\delta(D) \rightarrow 0} R(f, D, T_D).$$
Če zgornja limita $\exists$, rečemo, da je funkcija $f$ \textit{integrabilna} na $[a, b]$.

\end{definicija}
\vspace{0.5cm}

\begin{definicija}

$$\lim_{\delta(D) \rightarrow 0} R(f, D, T_D) = I,$$
če za $\forall \epsilon > 0 ~ \exists \delta > 0$, da za poljubno delitev $D$ z $\delta(D) < \delta$ in poljuben usklajen nabor testnih točk $T_D$ velja
$$|R(f, D, T_D) - I| < \epsilon.$$ 

\end{definicija}
\vspace{0.5cm}

% *************************************************************************************************

\subsection{Integrabilne funkcije}

\begin{definicija}[Zožitev]

Naj bo $f:A \rightarrow \mathbb{R}$ funkcija in $B \subset A$. Tedaj $f |_B: B \rightarrow \mathbb{R}$ označuje funkcijo z definicijskim območjem $B$, ki $\forall x \in B$ preslika v $f(x)$. Funkcijo $f |_B$ imenujemo \textit{zožitev} funkcije $f$ na $B$.   

\end{definicija}
\vspace{0.5cm}

\begin{definicija}[Enakomerna zveznost]

Naj bo $A \subseteq \mathbb{R}^n$. Funkcija $f: A \rightarrow \mathbb{R}$ je \textit{enakomerno zvezna} na $A$, če za $\forall \epsilon > 0 ~ \exists \delta = \delta_\epsilon > 0$, da za poljubna $x, y \in A$, ki zadoščata $|x-y| < \delta$, velja
$$|f(x) - f(y)| < \epsilon.$$ 

\end{definicija}
\vspace{0.5cm}

\begin{definicija}[Odsekoma zvezna funkcija]

Funkcija $f: J \rightarrow \mathbb{R}$, definirana na omejenem intervalu $J$, je \textit{odsekoma zvezna}, če je zvezna v $\forall$ točkah intervala razen morda v končno mnogo točkah, kjer ima skoke.

Funkcija f ima \textit{skos} v točki $c \in J$, če $f$ ni zvezna v $c$, ima pa (končno) levo in desno limito $c$ (če je $c$ krajišče intervala, zahtevamo le obstoj limite na tisti strani $c$, ki leži v $J$). 

\end{definicija}

\begin{posledica}

Če je $f:[a, b] \rightarrow \mathbb{R}$ \textit{odsekoma zvezna}, potem je \textit{integrabilna}. Vrednosti funkcije f v skokih ne vplivajo niti na integrabilnost niti na integral funkcije $f$ na $[a, b]$.

\end{posledica}
\vspace{0.5cm}

\begin{dogovor}
	\begin{itemize}
		\item Integral po izrojenemu intervalu $[a, a]$ je nič: 
		$$\int_{a}^{a} f(x) dx = 0.$$ \\
		\item Če je $a < b$, je 
		$$\int_{b}^{a} f(x) dx = - \int_{a}^{b} f(x) dx.$$
	\end{itemize}
\end{dogovor}
\vspace{0.5cm}

\begin{definicija}[Povprečna vrednost]

\textit{Povprečna vrednost} integrabilne funkcije $f$ na intervalu $[a, b]$ je 
$$\mu := \frac{1}{b-a} \int_{a}^{b} f(x) dx.$$

\end{definicija}
\vspace{0.5cm}

% *************************************************************************************************

\subsection{Osnovni izrek analize}
\vspace{0.5cm}

\begin{definicija}

Naj bo $f:[a, b] \rightarrow \mathbb{R}$ \textit{integrabilna} funkcija. Funkcijo $F:[a, b] \rightarrow \mathbb{R}$, definirano s predpisom
$$F(x) = \int_{a}^{x} f(t) dt,$$
imenujemo \textit{integral kot funkcija zgornje meje}.

\end{definicija}
\vspace{0.5cm}

% *************************************************************************************************

\subsection{Pravila za integriranje in Leibnizova formula}
\vspace{0.5cm}

% *************************************************************************************************

\subsection{Posplošeni integral na omejenem intervalu}
\vspace{0.5cm}

\begin{definicija}[Posplošeni integral]

Naj bo $f:(a, b] \rightarrow \mathbb{R}$ funkcija, ki je integrabilna na intervalu $[t, b]$ za $\forall t \in (a, b)$. Potem je \textit{posplošeni integral} funkcije $f$ na intervalu $[a, b]$
$$\int_{a}^{b} f(x) dx := \lim_{t \searrow a} \int_{t}^{b} f(x) dx,$$
če ta limita $\exists$.

Če limita $\exists$ , rečemo, da je $f$ \textit{posplošeno integrabilna} na $[a, b]$ in da je $\int_{a}^{b} f(x) dx$ \textit{konvergenten}, sicer pa rečemo, da je integral \textit{divergenten}.

\end{definicija}
\vspace{0.5cm}

% *************************************************************************************************

\subsection{Posplošeni integral na neomejenem intervalu}
\vspace{0.5cm}

\begin{definicija}[Posplošena integrabilnost]
	\begin{itemize}
		\item Naj bo $f:[a, \infty) \rightarrow \mathbb{R}$ integrabilna na $[a, s]$ za $\forall s > a$. Potem je \textit{posplošeni integral} funkcije $f$ na $[a, \infty)$
		$$\int_{a}^{\infty} f(x) dx := \lim_{s \rightarrow \infty} \int_{a}^{s} f(x) dx,$$
		če ta limita $\exists$. Če limita $\exists$, rečemo, da je posplošeni integral \textit{konvergenten}, sicer pa, da je \textit{divergenten}. \\
		\item Naj bo $f:(-\infty, b] \rightarrow \mathbb{R}$ integrabilna na $[t, b]$ za $\forall t < b$. Potem je \textit{posplošeni integral} funkcije $f$ na $(-\infty, b]$
		$$\int_{-\infty}^{b} f(x) dx := \lim_{t \rightarrow -\infty} \int_{t}^{b} f(x) dx,$$
		če ta limita $\exists$. Če limita $\exists$, rečemo, da je posplošeni integral \textit{konvergenten}, sicer pa, da je \textit{divergenten}. \\
		\item Funkcija $f:(-\infty, \infty) \rightarrow \mathbb{R}$ je \textit{posplošeno integrabilna}, če sta posplošeno integrabilni zožitvi $f |_{(-\infty, a]}$ in $f |_{[a, \infty)}$ za $\forall a \in \mathbb{R}$. 
	\end{itemize}
\end{definicija}
\vspace{0.5cm}

% *************************************************************************************************

\pagebreak

% #################################################################################################

\section{KRIVULJE V RAVNINI}
\vspace{0.5cm}

% *************************************************************************************************

\subsection{Podajanje krivulj}
\vspace{0.5cm}

\begin{itemize}
	\item \textsc{Eksplicitno:} Funkcija $f: j \rightarrow \mathbb{R}$ za $J \subseteq \mathbb{R}$ določa krivuljo $\Gamma_f$, ki je graf te funkcije, torej
	$$\Gamma_f = \{(x, f(x)) \mid x \in J\}.$$
	\item \textsc{Implicitno:} Funkcija $g: A \rightarrow \mathbb{R}$ za $A \subseteq \mathbb{R}^2$ določa krivuljo $K_g$, ki je množica rešitev enačbe $g(x, y) = 0$, torej
	$$K_g = \{(x, y) \in A \mid g(x, y) = 0\}.$$
	\item \textsc{Parametrično:} Funkciji $\alpha, \beta: J \rightarrow \mathbb{R}$ za $J \subseteq \mathbb{R}$ določata krivuljo $K_F$, ki je množica vseh točk $(x, y)$, določenih z $x = \alpha(t)$ in $y = \beta(t)$, torej
	$$K_F = \{(\alpha(t), \beta(t)) \mid J\}.$$
	Preslikavo $F: J \rightarrow \mathbb{R}^2$, $F(t) := (\alpha(t), \beta(t))$ imenujemo \textit{pot} ali \textit{parametrizacija} krivulje $K_F$. Krivuljo $K_F$ imenujemo tudi \textit{tir} poti $F$.
	\item \textsc{Polarno:} Funkcija $h: J \rightarrow \mathbb{R}$ za $J \subseteq \mathbb{R}$ določa krivuljo $K_h$, ki je množica točk v ravnini s polarnima koordinatama $(r, \theta)$, kjer je $r = h(\theta)$, torej
	$$K_h = \{(h(\theta)\cos{\theta}, h(\theta)\sin{\theta}) \mid \theta \in J\}.$$
\end{itemize}
\vspace{0.5cm}

% *************************************************************************************************

\subsection{Enačba tangente na krivuljo}
\vspace{0.5cm}

\begin{definicija}[Regularna točka]

Naj bo $g: A \rightarrow \mathbb{R}$ odvedljiva v točki $(a, b) \in A \subseteq \mathbb{R}^2$. Če je 
$$\nabla g(a, b) \neq (0, 0),$$
rečemo, da je $(a, b)$ \textit{regularna točka} za $g$, sicer pa, da je $(a, b)$ \textit{singularna točka} za $g$. 

\end{definicija}
\vspace{0.5cm}

\begin{definicija}

Naj bosta $\alpha, \beta: J \rightarrow \mathbb{R}$ \textit{odvedljivi}, kjer je $J \subseteq \mathbb{R}$ interval, ter $F = (\alpha, \beta)$ pripadajoča \textit{odvedljiva} pot. \textit{Odvod poti} $F$ po $t$ je hitrostni vektor $\dot{F}(t) = (\dot{\alpha}(t), \dot{\beta}(t))$. \\
Če je 
$$\dot{F}(t) \neq (0, 0)$$ 
za neki $t \in J$, imenujemo $t$ \textit{regularna točka} parametrizacije $F$. Če so $\forall$ točke intervala $J$ regularne, imenujemo $F$ \textit{regularna parametrizacija}. \\
Naj bo $g: I \rightarrow J$ \textit{odvedljiva surjektivna} funkcija, kjer je $I \subset \mathbb{R}$ interval. Pot
$$G := F \circ g$$
imenujemo \textit{reparametrizacija} poti $F$.

\end{definicija}
\vspace{0.5cm}

% *************************************************************************************************

\subsection{Dolžina loka krivulje}
\vspace{0.5cm}

\begin{definicija}

Naj bo dana pot $F:[a, b] \rightarrow \mathbb{R}^2$, $F(t) = (\alpha(t), \beta(t))$, ki določa krivuljo $K$. Izberimo delitev
$$D = \{a = t_0 < t_1 < \ldots < t_n = b\}$$
intervala $[a, b]$. Pot $F(t)$ na $i$-tem podintervalu $[t_{i-1}, t_i]$ zamenjamo z daljico od $F(t_{i-1})$ do $F(t_i)$. \\
Dolžina tako nastale lomljene črte, ki aproksimira tir poti $F$, je 
$$\ell(D) = \sum_{i = 1}^{n} \sqrt{ (\alpha(t_i) - \alpha(t_{i-1}))^2 + (\beta(t_i) - \beta(t_{i-1}))^2 }.$$
Če $\exists$ limita dolžin $\ell(D)$, ko pošljemo velikost delitve $\delta(D)$ proti nič (neodvisno od izbire delitev), jo imenujemo \textit{dolžina poti} $F$ in označimo $\ell(F)$:
$$\ell(F) = \lim_{D,\delta(D) \rightarrow 0} \ell(D).$$

\end{definicija}
\vspace{0.5cm}

\begin{definicija}[Ločna dolžina]

Diferencial dolžina loka krivulje označimo z $ds$ in ga imenujemo \textit{ločna dolžina}. V vseh opisih krivulje velja 
$$ds^2 = dx^2 + dy^2.$$

\end{definicija}
\vspace{0.5cm}

\begin{uporaba}[Površina rotacijske ploskve]

Naj bo $f:[a, b] \rightarrow \mathbb{R}$ \textit{nenegativna zvezna} funkcija. Ploskev, ki jo dobimo z vrtenjem grafa funkcije $f$ nad intervalom $[a, b]$ okoli osi $x$, imenujemo \textit{rotacijska ploskev}. \\

Izberemo neko delitev
$$D = \{a = x_0 < x_1 < \ldots < x_n = b\}$$
intervala $[a, b]$. Nad intervalom $[x_{i-1}, x_i]$ graf funkcije $f$ aproksimiramo z daljico od točke $(x_{i-1}, f(x_{i-1})$ do točke $(x_i, f(x_i))$. Ko daljico zavrtimo okoli $x$-osi, dobimo plašč prisekanega stožca s polmeroma leve in desne mejne krožnice $f(x_{i-1})$ in $f(x_i)$ ter višino $\delta_i = x_i - x_{i-1}$. To da približek za površino ploskve:
$$\sum_{i=1}^{n} \pi (f(x_{i-1}) + f(x_i)) \sqrt{{\delta_i}^2 + (f(x_{i-1}) - f(x_i))^2}.$$
Če je $f$ \textit{zvezno odvedljiva}, dobimo za površino v limiti, ko pošljemo velikost delitve $\delta(D)$ proti $0$, formulo
$$P = 2 \pi \int_{a}^{b} f(x) \sqrt{1 + (f'(x))^2} dx = 2 \pi \int_{a}^{b} y \sqrt{1 + {y'}^2} dx.$$

\end{uporaba}
\vspace{0.5cm}

% *************************************************************************************************

\subsection{Ploščina območja, določenega s krivuljo}
\vspace{0.5cm}

\begin{definicija}

Naj bo $F:[a, b] \rightarrow \mathbb{R}^2$ regularna parametrizacija krivulje K. Potem $F$ določa \textit{usmerjenost} K, določeno s smerjo, v kateri potuje točka $F(t)$ po $K$, ko potuje $t$ od $a$ do $b$. \\

\textit{Gladka enostavna sklenjena krivulja} je krivulja $K$, ki ima regularno parametrizacijo $F:[a, b] \rightarrow \mathbb{R}^2$, za katero velja $F(a) = F(b)$ in $\dot{F}(a) = \dot{F}(b)$, $F |_{[a, b)}$ pa je \textit{injektivna}. \\

Naj bo $A$ območje, ki ga omejuje \textit{gladka enostavna sklenjena krivulja} $K$. Regularna parametrizacija $F$ krivulje $K$ določa \textit{pozitivno usmerjenost} krivulje $K$, če je $A$ na levi strani, ko se pomikamo vzdolž $K$ v smeri usmerjenosti, ki jo določa $F$.

\end{definicija}
\vspace{0.5cm}

% *************************************************************************************************

\subsection{Diferencialne enačbe v obliki diferenciala}
\vspace{0.5cm}

\begin{definicija}

\textit{Diferencialna enačba v obliki diferenciala} je enačba oblike 
$$P(x, y)~dx + Q(x, y)~dy = 0,$$
kjer sta $P, Q: A \rightarrow \mathbb{R}$ definirani na nekem območju $A \subset \mathbb{R}^2$. \\

Naj bosta $P, Q: A \rightarrow \mathbb{R}$ \textit{odvedljivi}. Diferencialna enačba $P(x, y)~dx + Q(x, y)~dy = 0$ je \textit{eksaktna} na $A$, če velja 
$$\frac{\partial P}{\partial y}(x, y) = \frac{\partial Q}{\partial x}(x, y)$$
za $\forall x \in A$.

\end{definicija}
\vspace{0.5cm}

\begin{definicija}[Integral s parametrom]

Naj bo $A = [a, b] \times [c, d] \subset \mathbb{R}^2$ pravokotnik in naj bo $f: A \rightarrow \mathbb{R}$ \textit{zvezna} funkcija. Funkcijo $F:[a, b] \rightarrow \mathbb{R}$, definirano s predpisom 
$$F(x) = \int_{c}^{d} f(x, y) dy,$$
imenujemo \textit{integral s parametrom}.

\end{definicija}
\vspace{0.5cm}

\begin{definicija}[Integrirajoči množitelj]

Naj bo dana diferencialna enačba $P(x, y)~dx + Q(x, y)~dy = 0$, kjer sta $P, Q: A \rightarrow \mathbb{R}$ \textit{zvezno odvedljivi} funkciji. Če je $\mu: A \rightarrow \mathbb{R}$ takšna \textit{zvezno odvedljiva} funkcija, da je enačba
$$\mu(x, y) P(x, y)~dx + \mu(x, y) Q(x, y)~dy = 0$$
\textit{eksaktna}, potem funkcijo $\mu$ imenujemo \textit{integrirajoči množitelj} dane enačbe.

\end{definicija}
\vspace{0.5cm}

\end{document}