\documentclass[11pt]{article}
\usepackage[utf8]{inputenc}
\usepackage[slovene]{babel}

\usepackage{amsthm}
\usepackage{amsmath, amssymb, amsfonts}
\usepackage{relsize}
\usepackage{simpsons}
%\usepackage{geometry}
% \geometry{
% a4paper,
% total={170mm,257mm},
% left=20mm,
% top=20mm,
% }

\theoremstyle{definition}
\newtheorem{definicija}{Definicija}[section]

\theoremstyle{theorem}
\newtheorem{uporaba}{Uporaba}[section]

\newtheorem{lema}{Lema}
\newtheorem{trditev}{Trditev}
\newtheorem{izrek}{Izrek}
\newtheorem*{dokaz}{Dokaz}
\newtheorem*{posledica}{Posledica}
\newtheorem*{dogovor}{Dogovor}
\newtheorem*{opomba}{Opomba}

\title{Analiza 2 - definicije, trditve in izreki}
\author{\Left \Homer ~Oskar Vavtar \Bart}
\date{2020/21}

\begin{document}
\maketitle
\pagebreak
\tableofcontents
\pagebreak

% #################################################################################################

\section{NEDOLOČENI INTEGRAL IN POJEM \\ DIFERENCIALNE ENAČBE}
\vspace{0.5cm}

% *************************************************************************************************

\subsection{Primitivna funkcija in nedoločeni integral}
\vspace{0.5cm}

\begin{definicija}[Primitivna funkcija]

Naj bo $f$ funkcija \textit{ene spremenljivke}. Če obstaja \textit{odvedljiva funkcija} $F: A \rightarrow \mathbb{R}$, za katero velja $F' = f$, imenujemo F \textit{primitivna funkcija} funkcije $f$ na $A$. 

\end{definicija}
\vspace{0.5cm}

\begin{lema}

Naj bosta $F$ in $G$ \textit{primitivni funkciji} za funkcijo $f$ na nekem intervalu $J$. Potem obstaja konstanta $C \in \mathbb{R}$, da velja 
$$G(x) ~=~ F(x) + C ~~~\textit{za}~ \forall x \in J.$$

\end{lema}
\vspace{0.5cm}

\begin{definicija}[Nedoločeni integral]

\textit{Nedoločeni integral} funkcije $f$ je skupek vseh njenih primitivnih funkcij. Označimo ga z $\mathlarger{\int f(x) dx}$, funkcijo $f$ pa imenujemo \textit{integrand}.

\end{definicija}

\begin{posledica}

Naj bo $F$ neka \textit{primitivna} funkcija za $f$ na intervalu $J$. Potem je za $x \in J$
$$\int f(x) dx ~=~ F(x) + C,$$
kjer je $C \in \mathbb{R}$ poljubna konstanta, ki jo imenujemo \textit{splošna} ali \textit{integracijska konstanta}.

\end{posledica}
\vspace{0.5cm}

\begin{trditev}[Lastnosti nedoločenega integrala]

Za poljubni funkciji $f$ in $g$, ki imata primitivni funkciji na intervalu $J$, ter skalar $a \in \mathbb{R}$ velja
\begin{align*}
\int \left( f(x) \pm g(x) \right)~dx ~&=~ \int f(x)~dx \pm \int g(x)~dx \\
\int a f(x)~dx ~&=~ a \int f(x)~dx
\end{align*}
za $x \in J$; torej je nedoločeni integral \textit{linearen}. Če je $F$ odvedljiva na $J$, potem za $x \in J$ velja
$$\int F'(x) dx ~=~ F(x) + C,$$
kjer je $C \in \mathbb{R}$ poljubna konstanta.

\end{trditev}
\vspace{0.5cm}

% *************************************************************************************************

\subsection{Uvedba nove spremenljivke v nedoločeni integral}
\vspace{0.5cm}

\begin{trditev}

Naj bo funkcija $g$ \textit{odvedljiva} na intervalu $J$ in naj ima funkcija $f$ primitivno funkcijo $F$ na intervalu $g(J) = \{ g(x); ~x \in J \}$. Potem je $F \circ g$ \textit{primitivna} funkcija za $(f \circ g) \cdot g'$ na $J$, torej je
$$\int f \left( g(x) \right) g'~dx ~=~ \int f(t)~dt,$$
kjer smo s $t = g(x)$ označili novo spremenljivko.

\end{trditev}
\vspace{0.5cm}

% *************************************************************************************************

\subsection{Integracija po delih v nedoločenem integralu}
\vspace{0.5cm}

\begin{trditev}

Naj bosta $f$ in $g$ \textit{odvedljivi} funkciji na intervalu $J$. Potem velja
$$\int f(x) g'(x)~dx ~=~ f(x) g(x) - \int g(x) f'(x)~dx.$$
Če označimo $u = f(x)$ in $v = g(x)$, lahko zgornjo formulo krajše zapišemo kot
$$\int u~dv ~=~ uv - \int v~du.$$

\end{trditev}
\vspace{0.5cm}

\pagebreak

% *************************************************************************************************

\subsection{Diferencialne enačbe 1. reda}
\vspace{0.5cm}

\begin{definicija}

\textit{Navadna diferencialna enačba} 1. reda je enačba za neznano funkcijo 
$$y=g(x),$$
ki vsebuje tudi odvod $y'$ funkcije $y$. \\

\noindent \textit{Splošna oblika} diferencialne enačbe 1. reda je
$$F(x,y,y')=0,$$
kjer je $F$ funkcija treh spremenljivk, ki je res odvisna od zadnje spremenljivke. \\

\noindent Če iz diferencialne enačbe izrazimo odvod funkcije, dobimo \textit{standardno obliko} diferencialne enačbe 1. reda
$$y' ~=~ f(x,y).$$

\noindent Začetni pogoj za diferencialno enačbo 1. reda v točki $x = a$ je podan z vrednostjo iskane funkcije $y$ v točki $a$, torej $y(a) = b$, kjer je $b \in \mathbb{R}$. Diferencialna enačba in začetni pogoj skupaj sestavljata začetno nalogo.

\end{definicija}
\vspace{0.5cm}

\begin{definicija}

\textit{Eksplicitna rešitev} diferencialne enačbe $F(x,y,y') = 0$ je takšna odvedljiva funkcija $g:J \rightarrow R$, definirana na intervalu $J$, da postane diferencialna enačba identiteta na $J$, le vanjo vstavimo $y = g(x)$, torej za $\forall x \in J$ velja 
$$F(x,g(x),g'(x)) ~=~ 0.$$

\noindent \textit{Implicitna rešitev} diferencialne enačbe je dana z enačbo $G(x,y) = 0$, ki na nekem intervalu določa eksplicitno rešitev dane diferencialne enačbe. \\

\noindent \textit{Splošna rešitev} diferencialne enačbe 1. reda je funkcija 
$$y=g(x,C)~\footnote{\text{Lahko podana implicitno.}},$$ ki je odvisna od splošne konstante $C$ in reši dano diferencialno enačbo za poljubno izbiro vrednosti konstante $C \in \mathbb{R}$, poleg tega pa za poljuben začetni pogoj obstaja vrednost konstante $C$, pri kateri rešitev zadošča izbranemu začetnemu pogoju. Rešitev, ki ne vsebuje splošnih konstant, imenujemo tudi \textit{posebna} ali \textit{partikularna} rešitev.

\end{definicija}
\vspace{0.5cm}

\begin{definicija}[LDE 1. reda]

\textit{Linearna diferencialna enačba} 1. reda ima obliko
$$r_1(x)y' + r_0(x)y = s(x),$$
kjer so $r_0, r_1, s: J \rightarrow \mathbb{R}$ funkcije, definirane na nekem intervalu $J$. Če je $s$ ničelna funkcija, rečemo, da je enačba \textit{homogena}. Če sta funkciji $r_0$, $r_1$ \textit{konstantni}, pa rečemo, da ima enačba \textit{konstante koeficiente}. \\

\noindent \textit{Standardna oblika} linearne diferencialne enačbe 1. reda je
$$y' + p(x)y = q(x),$$
kjer sta $p, q: J \rightarrow \mathbb{R}$ funkciji, definirani na intervalu $J$.

\end{definicija}
\vspace{0.5cm}

\begin{trditev}

Naj bo dana \textit{linearna} diferencialna enačba 
$$y' + p(x) y ~=~ q(x),$$
kjer sta $p, q: J \rightarrow \mathbb{R}$ \textit{zvezni} funkciji, definirani na intervalu $J$.
\begin{enumerate}

	\item[(i)] Splošna rešitev pripadajoče homogene diferencialne enačbe \\ $y' + p(x) y = 0$ je
	$$y(x) ~=~ Ce^{-\int p(x)~dx},$$
	kjer je $C \in \mathbb{R}$ splošna konstanta, v kateri je zajeta integracijska konstanta.
	
	\item[(ii)] Splošna rešitev dane diferencialne enačbe dobimo iz splošne rešitve pripadajoče homogene enačbe z variacijo konstante, torej z nastavkom
	$$y(x) ~=~ C(x)e^{-\int p(x)~dx},$$
	kjer funkcija $C(x)$ zadošča
	$$C'(x) ~=~ g(x)e^{\int p(x)~dx}.$$

\end{enumerate}

\end{trditev}
\vspace{0.5cm}

% *************************************************************************************************

\pagebreak

% #################################################################################################

\section{DOLOČENI INTEGRAL}
\vspace{0.5cm}

% *************************************************************************************************

\subsection{Motivacija za določeni integral}
\vspace{0.5cm}

\begin{definicija}

Naj bo $f:[a, b] \rightarrow \mathbb{R}$ \textit{nenegativna funkcija}, torej $f(x) \geq 0$ za vse $x \in [a, b]$. Rečemo, da graf funkicje f \textit{določa območje} $A \subset \mathbb{R}^2$  nad intervalom $[a, b]$. Množica $A$ je navzgor omejena z grafom funkcije $f$, na levi s premico $x=a$ in na desni s premico $x=b$.

\end{definicija}
\vspace{0.5cm}

% *************************************************************************************************

\subsection{Riemannova vsota in Riemannov integral}
\vspace{0.5cm}

\begin{definicija}[Riemannova vsota]

\textit{Delitev} $D$ intervala $[a, b]$ na podintervale je dana z izbiro \textit{delilnih točk} $x_i$:
$$a=x_0 < x_1 < x_2 < \ldots < x_{n-1} < x_n=b,$$
kjer je $n \in \mathbb{N}$. Dolžino $i$-tega podintervala $[x_{i-1},x_i]$ (za $i=1,2,...,n$) označimo z $\delta_i := x_i - x_{i-1}$. \textit{Velikost delitve} $D$ je dolžina najdaljšega podintervala delitve $D$, torej
$$\delta(D) = \max{\{\delta_i \mid i = 1, 2, ..., n\}}.$$

Na vsakem od podintervalov, na katere delitev $D$ razdeli interval $[a, b]$, izberemo \textit{testno točko} $t_i \in [x_{i-1}, x_i]$ in s $T_D = (t_1, t_2, \dots, t_n)$ označimo nabor teh točk; nabor testnih točk je \textit{usklajen} z delitvijo $D$, ker smo na vsakem podintervalu $[x_{i-1},x_i]$, določenem z $D$, izbrali natanko eno testno točko $t_i$.

\textit{Riemannova vsota} funkcije $f:[a, b] \rightarrow \mathbb{R}$, pridružena delitvi $D$ in usklajenemu naboru testnih točk $T_D$ je 
$$R(f, D, T_D) := \sum_{i=1}^{n} f(t_i) \delta_i.$$

\end{definicija}
\vspace{0.5cm}

\begin{definicija}[Riemannov integral]

\textit{Riemannov integral} ali \textit{določeni integral} funkcije $f:[a, b] \rightarrow \mathbb{R}$ je limita Riemannovih vsot $R(f, D, T_D)$, kjer limito vzamemo po vseh delitvah D intervala [a, b] in usklajenih naborih testnih točk $T_D$, ko pošljemo velikost delitev $\delta(D)$ proti $0$, če ta limita obstaja (torej je končna in neodvisna od izbire delitev in testnih točk). Pišemo
$$\int_{a}^{b} f(x) dx := \lim_{\delta(D) \rightarrow 0} R(f, D, T_D).$$
Če zgornja limita obstaja, rečemo, da je funkcija $f$ \textit{integrabilna} na $[a, b]$.

\end{definicija}
\vspace{0.5cm}

\begin{definicija}

$$\lim_{\delta(D) \rightarrow 0} R(f, D, T_D) = I,$$
če za $\forall \varepsilon > 0 ~ \exists \delta > 0$, da za poljubno delitev $D$ z $\delta(D) < \delta$ in poljuben usklajen nabor testnih točk $T_D$ velja
$$|R(f, D, T_D) - I| < \varepsilon.$$ 

\end{definicija}
\vspace{0.5cm}

% *************************************************************************************************

\subsection{Integrabilne funkcije}

\begin{definicija}[Zožitev]

Naj bo $f:A \rightarrow \mathbb{R}$ funkcija in $B \subset A$. Tedaj $f |_B: B \rightarrow \mathbb{R}$ označuje funkcijo z definicijskim območjem $B$, ki $\forall x \in B$ preslika v $f(x)$. Funkcijo $f |_B$ imenujemo \textit{zožitev} funkcije $f$ na $B$.   

\end{definicija}
\vspace{0.5cm}

\begin{definicija}[Enakomerna zveznost]

Naj bo $A \subseteq \mathbb{R}^n$. Funkcija $f: A \rightarrow \mathbb{R}$ je \textit{enakomerno zvezna} na $A$, če za $\forall \varepsilon > 0 ~ \exists \delta = \delta_\varepsilon > 0$, da za poljubna $x, y \in A$, ki zadoščata $|x-y| < \delta$, velja
$$|f(x) - f(y)| < \varepsilon.$$ 

\end{definicija}
\vspace{0.5cm}

\begin{definicija}[Odsekoma zvezna funkcija]

Funkcija $f: J \rightarrow \mathbb{R}$, definirana na omejenem intervalu $J$, je \textit{odsekoma zvezna}, če je zvezna v vseh točkah intervala razen morda v končno mnogo točkah, kjer ima skoke.

Funkcija f ima \textit{skos} v točki $c \in J$, če $f$ ni zvezna v $c$, ima pa (končno) levo in desno limito $c$ (če je $c$ krajišče intervala, zahtevamo le obstoj limite na tisti strani $c$, ki leži v $J$). 

\end{definicija}

\begin{posledica}

Če je $f:[a, b] \rightarrow \mathbb{R}$ \textit{odsekoma zvezna}, potem je \textit{integrabilna}. Vrednosti funkcije f v skokih ne vplivajo niti na integrabilnost niti na integral funkcije $f$ na $[a, b]$.

\end{posledica}
\vspace{0.5cm}

\begin{dogovor}
~
	\begin{itemize}
		\item Integral po izrojenemu intervalu $[a, a]$ je nič: 
		$$\int_{a}^{a} f(x) dx = 0.$$ \\
		\item Če je $a < b$, je 
		$$\int_{b}^{a} f(x) dx = - \int_{a}^{b} f(x) dx.$$
	\end{itemize}
\end{dogovor}
\vspace{0.5cm}

\begin{definicija}[Povprečna vrednost]

\textit{Povprečna vrednost} integrabilne funkcije $f$ na intervalu $[a, b]$ je 
$$\mu := \frac{1}{b-a} \int_{a}^{b} f(x) dx.$$

\end{definicija}
\vspace{0.5cm}

% *************************************************************************************************

\subsection{Osnovni izrek analize}
\vspace{0.5cm}

\begin{definicija}

Naj bo $f:[a, b] \rightarrow \mathbb{R}$ \textit{integrabilna} funkcija. Funkcijo $F:[a, b] \rightarrow \mathbb{R}$, definirano s predpisom
$$F(x) = \int_{a}^{x} f(t) dt,$$
imenujemo \textit{integral kot funkcija zgornje meje}.

\end{definicija}
\vspace{0.5cm}

% *************************************************************************************************

\subsection{Pravila za integriranje in Leibnizova formula}
\vspace{0.5cm}

% *************************************************************************************************

\subsection{Posplošeni integral na omejenem intervalu}
\vspace{0.5cm}

\begin{definicija}[Posplošeni integral]

Naj bo $f:(a, b] \rightarrow \mathbb{R}$ funkcija, ki je integrabilna na intervalu $[t, b]$ za $\forall t \in (a, b)$. Potem je \textit{posplošeni integral} funkcije $f$ na intervalu $[a, b]$
$$\int_{a}^{b} f(x) dx := \lim_{t \searrow a} \int_{t}^{b} f(x) dx,$$
če ta limita obstaja.

Če limita obstaja , rečemo, da je $f$ \textit{posplošeno integrabilna} na $[a, b]$ in da je $\mathlarger{\int_{a}^{b} f(x) dx}$ \textit{konvergenten}, sicer pa rečemo, da je integral \textit{divergenten}.

\end{definicija}
\vspace{0.5cm}

% *************************************************************************************************

\subsection{Posplošeni integral na neomejenem intervalu}
\vspace{0.5cm}

\begin{definicija}[Posplošena integrabilnost]
	\begin{itemize}
		\item Naj bo $f:[a, \infty) \rightarrow \mathbb{R}$ integrabilna na $[a, s]$ za $\forall s > a$. Potem je \textit{posplošeni integral} funkcije $f$ na $[a, \infty)$
		$$\int_{a}^{\infty} f(x) dx := \lim_{s \rightarrow \infty} \int_{a}^{s} f(x) dx,$$
		če ta limita obstaja. Če limita obstaja, rečemo, da je posplošeni integral \textit{konvergenten}, sicer pa, da je \textit{divergenten}. \\
		\item Naj bo $f:(-\infty, b] \rightarrow \mathbb{R}$ integrabilna na $[t, b]$ za $\forall t < b$. Potem je \textit{posplošeni integral} funkcije $f$ na $(-\infty, b]$
		$$\int_{-\infty}^{b} f(x) dx := \lim_{t \rightarrow -\infty} \int_{t}^{b} f(x) dx,$$
		če ta limita obstaja. Če limita obstaja, rečemo, da je posplošeni integral \textit{konvergenten}, sicer pa, da je \textit{divergenten}. \\
		\item Funkcija $f:(-\infty, \infty) \rightarrow \mathbb{R}$ je \textit{posplošeno integrabilna}, če sta posplošeno integrabilni zožitvi $f |_{(-\infty, a]}$ in $f |_{[a, \infty)}$ za $\forall a \in \mathbb{R}$. 
	\end{itemize}
\end{definicija}
\vspace{0.5cm}

% *************************************************************************************************

\pagebreak

% #################################################################################################

\section{KRIVULJE V RAVNINI}
\vspace{0.5cm}

% *************************************************************************************************

\subsection{Podajanje krivulj}
\vspace{0.5cm}

\begin{itemize}
	\item \textsc{Eksplicitno:} Funkcija $f: j \rightarrow \mathbb{R}$ za $J \subseteq \mathbb{R}$ določa krivuljo $\Gamma_f$, ki je graf te funkcije, torej
	$$\Gamma_f = \{(x, f(x)) \mid x \in J\}.$$
	\item \textsc{Implicitno:} Funkcija $g: A \rightarrow \mathbb{R}$ za $A \subseteq \mathbb{R}^2$ določa krivuljo $K_g$, ki je množica rešitev enačbe $g(x, y) = 0$, torej
	$$K_g = \{(x, y) \in A \mid g(x, y) = 0\}.$$
	\item \textsc{Parametrično:} Funkciji $\alpha, \beta: J \rightarrow \mathbb{R}$ za $J \subseteq \mathbb{R}$ določata krivuljo $K_F$, ki je množica vseh točk $(x, y)$, določenih z $x = \alpha(t)$ in $y = \beta(t)$, torej
	$$K_F = \{(\alpha(t), \beta(t)) \mid J\}.$$
	Preslikavo $F: J \rightarrow \mathbb{R}^2$, $F(t) := (\alpha(t), \beta(t))$ imenujemo \textit{pot} ali \textit{parametrizacija} krivulje $K_F$. Krivuljo $K_F$ imenujemo tudi \textit{tir} poti $F$.
	\item \textsc{Polarno:} Funkcija $h: J \rightarrow \mathbb{R}$ za $J \subseteq \mathbb{R}$ določa krivuljo $K_h$, ki je množica točk v ravnini s polarnima koordinatama $(r, \theta)$, kjer je $r = h(\theta)$, torej
	$$K_h = \{(h(\theta)\cos{\theta}, h(\theta)\sin{\theta}) \mid \theta \in J\}.$$
\end{itemize}
\vspace{0.5cm}

% *************************************************************************************************

\subsection{Enačba tangente na krivuljo}
\vspace{0.5cm}

\begin{definicija}[Regularna točka]

Naj bo $g: A \rightarrow \mathbb{R}$ odvedljiva v točki $(a, b) \in A \subseteq \mathbb{R}^2$. Če je 
$$\nabla g(a, b) \neq (0, 0),$$
rečemo, da je $(a, b)$ \textit{regularna točka} za $g$, sicer pa, da je $(a, b)$ \textit{singularna točka} za $g$. 

\end{definicija}
\vspace{0.5cm}

\begin{definicija}

Naj bosta $\alpha, \beta: J \rightarrow \mathbb{R}$ \textit{odvedljivi}, kjer je $J \subseteq \mathbb{R}$ interval, ter $F = (\alpha, \beta)$ pripadajoča \textit{odvedljiva} pot. \textit{Odvod poti} $F$ po $t$ je hitrostni vektor $\dot{F}(t) = (\dot{\alpha}(t), \dot{\beta}(t))$. \\
Če je 
$$\dot{F}(t) \neq (0, 0)$$ 
za neki $t \in J$, imenujemo $t$ \textit{regularna točka} parametrizacije $F$. Če so vse točke intervala $J$ regularne, imenujemo $F$ \textit{regularna parametrizacija}. \\
Naj bo $g: I \rightarrow J$ \textit{odvedljiva surjektivna} funkcija, kjer je $I \subset \mathbb{R}$ interval. Pot
$$G := F \circ g$$
imenujemo \textit{reparametrizacija} poti $F$.

\end{definicija}
\vspace{0.5cm}

% *************************************************************************************************

\subsection{Dolžina loka krivulje}
\vspace{0.5cm}

\begin{definicija}

Naj bo dana pot $F:[a, b] \rightarrow \mathbb{R}^2$, $F(t) = (\alpha(t), \beta(t))$, ki določa krivuljo $K$. Izberimo delitev
$$D = \{a = t_0 < t_1 < \ldots < t_n = b\}$$
intervala $[a, b]$. Pot $F(t)$ na $i$-tem podintervalu $[t_{i-1}, t_i]$ zamenjamo z daljico od $F(t_{i-1})$ do $F(t_i)$. \\
Dolžina tako nastale lomljene črte, ki aproksimira tir poti $F$, je 
$$\ell(D) = \sum_{i = 1}^{n} \sqrt{ (\alpha(t_i) - \alpha(t_{i-1}))^2 + (\beta(t_i) - \beta(t_{i-1}))^2 }.$$
Če obstaja limita dolžin $\ell(D)$, ko pošljemo velikost delitve $\delta(D)$ proti nič (neodvisno od izbire delitev), jo imenujemo \textit{dolžina poti} $F$ in označimo $\ell(F)$:
$$\ell(F) = \lim_{D,\delta(D) \rightarrow 0} \ell(D).$$

\end{definicija}
\vspace{0.5cm}

\begin{definicija}[Ločna dolžina]

Diferencial dolžina loka krivulje označimo z $ds$ in ga imenujemo \textit{ločna dolžina}. V vseh opisih krivulje velja 
$$ds^2 = dx^2 + dy^2.$$

\end{definicija}
\vspace{0.5cm}

\begin{uporaba}[Površina rotacijske ploskve]

Naj bo $f:[a, b] \rightarrow \mathbb{R}$ \textit{nenegativna zvezna} funkcija. Ploskev, ki jo dobimo z vrtenjem grafa funkcije $f$ nad intervalom $[a, b]$ okoli osi $x$, imenujemo \textit{rotacijska ploskev}. \\

Izberemo neko delitev
$$D = \{a = x_0 < x_1 < \ldots < x_n = b\}$$
intervala $[a, b]$. Nad intervalom $[x_{i-1}, x_i]$ graf funkcije $f$ aproksimiramo z daljico od točke $(x_{i-1}, f(x_{i-1})$ do točke $(x_i, f(x_i))$. Ko daljico zavrtimo okoli $x$-osi, dobimo plašč prisekanega stožca s polmeroma leve in desne mejne krožnice $f(x_{i-1})$ in $f(x_i)$ ter višino $\delta_i = x_i - x_{i-1}$. To da približek za površino ploskve:
$$\sum_{i=1}^{n} \pi (f(x_{i-1}) + f(x_i)) \sqrt{{\delta_i}^2 + (f(x_{i-1}) - f(x_i))^2}.$$
Če je $f$ \textit{zvezno odvedljiva}, dobimo za površino v limiti, ko pošljemo velikost delitve $\delta(D)$ proti $0$, formulo
$$P = 2 \pi \int_{a}^{b} f(x) \sqrt{1 + (f'(x))^2} dx = 2 \pi \int_{a}^{b} y \sqrt{1 + {y'}^2} dx.$$

\end{uporaba}
\vspace{0.5cm}

% *************************************************************************************************

\subsection{Ploščina območja, določenega s krivuljo}
\vspace{0.5cm}

\begin{definicija}

Naj bo $F:[a, b] \rightarrow \mathbb{R}^2$ regularna parametrizacija krivulje K. Potem $F$ določa \textit{usmerjenost} K, določeno s smerjo, v kateri potuje točka $F(t)$ po $K$, ko potuje $t$ od $a$ do $b$. \\

\textit{Gladka enostavna sklenjena krivulja} je krivulja $K$, ki ima regularno parametrizacijo $F:[a, b] \rightarrow \mathbb{R}^2$, za katero velja $F(a) = F(b)$ in $\dot{F}(a) = \dot{F}(b)$, $F |_{[a, b)}$ pa je \textit{injektivna}. \\

Naj bo $A$ območje, ki ga omejuje \textit{gladka enostavna sklenjena krivulja} $K$. Regularna parametrizacija $F$ krivulje $K$ določa \textit{pozitivno usmerjenost} krivulje $K$, če je $A$ na levi strani, ko se pomikamo vzdolž $K$ v smeri usmerjenosti, ki jo določa $F$.

\end{definicija}
\vspace{0.5cm}

% *************************************************************************************************

\subsection{Diferencialne enačbe v obliki diferenciala}
\vspace{0.5cm}

\begin{definicija}

\textit{Diferencialna enačba v obliki diferenciala} je enačba oblike 
$$P(x, y)~dx + Q(x, y)~dy = 0,$$
kjer sta $P, Q: A \rightarrow \mathbb{R}$ definirani na nekem območju $A \subset \mathbb{R}^2$. \\

Naj bosta $P, Q: A \rightarrow \mathbb{R}$ \textit{odvedljivi}. Diferencialna enačba $P(x, y)~dx + Q(x, y)~dy = 0$ je \textit{eksaktna} na $A$, če velja 
$$\frac{\partial P}{\partial y}(x, y) = \frac{\partial Q}{\partial x}(x, y)$$
za $\forall x \in A$.

\end{definicija}
\vspace{0.5cm}

\begin{definicija}[Integral s parametrom]

Naj bo $A = [a, b] \times [c, d] \subset \mathbb{R}^2$ pravokotnik in naj bo $f: A \rightarrow \mathbb{R}$ \textit{zvezna} funkcija. Funkcijo $F:[a, b] \rightarrow \mathbb{R}$, definirano s predpisom 
$$F(x) = \int_{c}^{d} f(x, y) dy,$$
imenujemo \textit{integral s parametrom}.

\end{definicija}
\vspace{0.5cm}

\begin{definicija}[Integrirajoči množitelj]

Naj bo dana diferencialna enačba $P(x, y)~dx + Q(x, y)~dy = 0$, kjer sta $P, Q: A \rightarrow \mathbb{R}$ \textit{zvezno odvedljivi} funkciji. Če je $\mu: A \rightarrow \mathbb{R}$ takšna \textit{zvezno odvedljiva} funkcija, da je enačba
$$\mu(x, y) P(x, y)~dx + \mu(x, y) Q(x, y)~dy = 0$$
\textit{eksaktna}, potem funkcijo $\mu$ imenujemo \textit{integrirajoči množitelj} dane enačbe.

\end{definicija}
\vspace{0.5cm}

% *************************************************************************************************

\pagebreak

% #################################################################################################

\section{ŠTEVILSKE VRSTE}
\vspace{0.5cm}

% *************************************************************************************************

\subsection{Osnovni pojmi}
\vspace{0.5cm}

\begin{definicija}[Številska vrsta]

\textit{Številska vrsta} je vsote (neskončnega) zaporedja realnih števil. Če je $(a_n)_{n \in \mathbb{N}}$ zaporedje realnih števil, je pripadajoča številska vrsta
$$\sum_{n=1}^{\infty} a_n = a_1 + a_2 + \ldots + a_n + \ldots .$$
Za naravno število $k$ je \textit{$k$-ta delna vsota} vrste $\mathlarger{\sum_{n=1}^{\infty} a_n}$ enak
$$s_k := \sum_{n=1}^{k} a_n = a_1 + a_2 + \ldots + a_k.$$
Vrsta $\mathlarger{\sum_{n=1}^{\infty} a_n}$ je \textit{konvergentna (divergentna)}, če je konvergentno (divergentno) zaporedje delnih vsot $(s_k)_{k \in \mathbb{N}}$. Če je vrsta konvergentna, za njeno vsoto vzamemo limito delnih vsot.

\end{definicija}
\vspace{0.5cm}

% *************************************************************************************************

\subsection{Vrste s pozitivnimi členi}
\vspace{0.5cm}

% *************************************************************************************************

\subsection{Alternirajoče vrste}
\vspace{0.5cm}

\begin{definicija}[Alternirajoča vrsta]

Vrsta $\mathlarger{\sum_{n=1}^{\infty} a_n}$ je \textit{alternirajoča}, če je člen $a_{n+1}$ nasprotno predznačen kot člen $a_n$ za $\forall n \in \mathbb{N}$.

\end{definicija}
\vspace{0.5cm}

% *************************************************************************************************

\subsection{Absolutna konvergenca}

\begin{definicija}

Vrsta $\mathlarger{\sum_{n=1}^{\infty} a_n}$ je \textit{absolutno konvergentna}, če je konvergentna vrsta $\mathlarger{\sum_{n=1}^{\infty} |a_n|}$ iz absolutnih vrednosti členov vrste. \\
Če je vrsta $\mathlarger{\sum_{n=1}^{\infty} a_n}$ konvergentna, ni pa absolutno konvergenta, rečemo, da je \textit{pogojno konvergentna}.

\end{definicija}
\vspace{0.5cm}

% *************************************************************************************************

\pagebreak

% #################################################################################################

\section{FUNKCIJSKA ZAPOREDJA IN FUNKCIJSKE VRSTE}
\vspace{0.5cm}

% *************************************************************************************************

\subsection{Konvergenca funkcijskih zaporedij}
\vspace{0.5cm}

\begin{definicija}[Funkcijsko zaporedje]

Naj bo $A \subset \mathbb{R}$ in naj bo za $\forall n \in \mathbb{N}$ dana funkcija $f_n: A \rightarrow \mathbb{R}$. Tedaj funkcije $f_n$ sestavljajo \textit{funkcijsko zaporedje} $(f_n)_{n \in \mathbb{N}}$. \\

Če za $\forall a \in A$ obstaja limita številskega zaporedja $(f_n(a))_n$, rečemo, da funkcijsko zaporedje \textit{konvergira po točkah} na $A$. V tem primeru za $a \in A$ označimo 
$$f(a) := \lim_{n \rightarrow \infty} f_n(a).$$
Tako dobljeno funkcijo $f: A \rightarrow \mathbb{R}$ imenujemo \textit{limitna funkcija} zaporedja $(f_n)_n$; pišemo 
$$f = \lim_{n \rightarrow \infty} f_n$$
in rečemo, da zaporedje $(f_n)_n$ \textit{konvergira k $f$ po točkah}.

\end{definicija}
\vspace{0.5cm}

\begin{opomba}

Funkcijsko zaporedje $(f_n: A \rightarrow \mathbb{R})_n$ v splošnem konvergira le v točkah iz neke podmnožice $B$ definicijskega območja $A$; množico $B$ imenujemo \textit{konvergenčno območje} funkcijskega zaporedja.

\end{opomba}
\vspace{0.5cm}

\begin{definicija}[Enakomerna konvergenca funkcijskega zaporedja]

Naj funkcijsko zaporedje \\ $(f_n: A \rightarrow \mathbb{R})_n$ konvergira po točkah proti limitni funkciji $f: A \rightarrow \mathbb{R}$, torej za $\forall x \in A$ in za $\forall \varepsilon > 0$ \\ $\exists n_{x,\varepsilon}$, da za $\forall n \geq n_{x,\varepsilon}$ velja 
$$|f_n(x) - f(x)| < \varepsilon.$$
Pravimo, da zaporedje $(f_n)_n$ \textit{konvergira proti $f$ enakomerno na $A$}, če za $\forall  \varepsilon > 0 ~\exists n_{\varepsilon}$, da za $\forall n \geq n_{\varepsilon}$ in za $\forall x \in A$ velja
$$|f_n(x) - f(x)| < \varepsilon.$$

\end{definicija}
\vspace{0.5cm}

\begin{definicija}[Funkcijska vrsta]

Naj bo dano zaporedje funkcij $(f_n: A \rightarrow \mathbb{R})_n$. Vsoto $\mathlarger{\sum_{n=1}^{\infty} f_n}$ \\ imenujemo \textit{funkcijska vrsta}. \\

Funkcijska vrsta $\mathlarger{\sum_{n=1}^{\infty} f_n}$ \textit{konvergira po točkah na $A$}, če zaporedje delnih vsot $s_k = \mathlarger{\sum_{n=1}^{k} f_n}$ \\ konvergira po točah na A; to pomeni, da za $\forall a \in A$ konvergira številska vrsta $\mathlarger{\sum_{n=1}^{\infty} f_n(a)}.$ \\

Naj bo $s: A \rightarrow \mathbb{R}$ limitna funkcija zaporedja delnih vsot $(s_k)_k$. \\
Funkcijska vrsta $\mathlarger{\sum_{n=1}^{\infty}f_n}$ \textit{konvergira k $s: A \rightarrow \mathbb{R}$ enakomerno na A}, če zaporedje delnih vsot $(s_k)_k$ konvergira k $s$ enakomerno na $A$.
 

\end{definicija}
\vspace{0.5cm}

\begin{posledica}

Če so funkcije $(f_n: A \rightarrow \mathbb{R})_n$ zvezne in funkcijska vrsta $\mathlarger{\sum_{n=1}^{\infty} f_n}$ konvergira k $s: A \rightarrow \mathbb{R}$ enakomerno na $A$, potem je $s$ zvezna na $A$.

\end{posledica}
\vspace{0.5cm}

% *************************************************************************************************

\subsection{Enakomerno konvergentne vrste}
\vspace{0.5cm}

% *************************************************************************************************

\subsection{Potenčne vrste}
\vspace{0.5cm}

\begin{definicija}[Potenčna vrsta]

Naj bo $(a_n)_{n \geq 0}$ \textit{realno zaporedje} in $c \in \mathbb{R}$. Funkcijsko zaporedje $$\sum_{n=0}^{\infty} a_n (x - c)^n$$
imenujemo \textit{potenčna vrsta s središčem v $c$}.

\end{definicija}
\vspace{0.5cm}

\begin{opomba}
~
\begin{enumerate}
	\item Potenčna vrsta vedno konvergira v središču $c$.
	\item Središče $c$ lahko vedno prestavimo v $0$ z uvedbo nove spremenljivke $t = x - c$:
	$$\sum_{n=0}^{\infty} a_n (x-c)^n = \sum_{n=0}^{\infty} a_n t^n.$$
\end{enumerate}
\end{opomba}
\vspace{0.5cm}

\begin{definicija}[Konvergenčni polmer]

Naj bo dana potenčna vrsta $\mathlarger{\sum_{n=0}^{\infty} a_n x^n}$. Število
$$R := \sup \{b \geq 0 \mid \text{vrsta konvergira pri } b \} \in [0, \infty)$$
imenujemo \textit{konvergenčni polmer} potenčne vrste. \\
Konvergenčno območje potenčne vrste označimo z $D$.

\end{definicija}
\vspace{0.5cm}

\begin{posledica}

Naj bo $R > 0$ \textit{konvergenčni polmer potenčne vrste} $\mathlarger{\sum_{n=0}^{\infty} a_n x^n}$. Potem vrsta \textit{absolutno konvergira} za $|x| < R$. Vrsta enakomerno konvergira na vsakem manjšem intervalu $|x| \leq r < R$. Vsota potenčne vrste je zvezna funkcija na $(-R, R)$. Vrsta divergira za $\forall x$, $|x| > R$; v krajiščih intervala lahko potenčna vrsta konvergira ali divergira. Torej je
$$(-R, R) \subseteq D \subseteq [-R, R].$$

\end{posledica}
\vspace{0.5cm}

\begin{definicija}[Limes superior]

Naj bo $(b_n)_n$ \textit{zaporedje realnih števil}. Največje stekališče zaporedja $(b_n)_n$ označimo 
$$\lim_{n \rightarrow \infty} \sup b_n \in [-\infty, \infty]$$
in ga imenujemo \textit{limes superior}.

\end{definicija}
\vspace{0.5cm}

% *************************************************************************************************

\subsection{Taylorjeva vrsta in analitične funkcije}
\vspace{0.5cm}

\begin{definicija}[Taylorjeva vrsta]

Naj bo $f$ \textit{poljubno mnogokrat zvezno odvedljiva} v okolici točke $c \in \mathbb{R}$. \textit{Taylorjeva vrsta funkcije $f$ s središčem v $c$} je 
$$T_c(x) = \sum_{n=0}^{\infty} \frac{f^{(n)}(c)}{n!} (x-c)^n.$$ \\

Naj bo $J \subset \mathbb{R}$ \textit{odprt interval} in $f \in \mathcal{C}^{\infty}(J)$. Rečemo, da je $f$ \textit{analitična} na $J$, če je za $\forall c \in J$ Taylorjeva vrsta $T_e$ enaka funkciji $f$ na neki okolici točke $c$.

\end{definicija}
\vspace{0.5cm}

% *************************************************************************************************

\subsection{Fourierove vrste}
\vspace{0.5cm}

\begin{definicija}[Fourierova vrsta]

\textit{Fourierova vrsta} je funkcijska vrsta oblike
$$a_0 + \sum_{n=1}^{\infty} (a_n \cos(nx) + b_n \sin(nx)),$$
kjer sta $(a_n)_{n \geq 0}$ in $(b_n)_{n \geq 1}$ \textit{realni} zaporedji.

\end{definicija}
\vspace{0.5cm}

\begin{definicija}[Skalarni produkt]

Naj bosta $f, g: [a, b] \rightarrow \mathbb{R}$ odekoma zvezni funkciji. \\ Izraz
$$\langle f, g \rangle := \int_{a}^{b} f(x) g(x) dx$$
imenujemo \textit{skalarni produkt} funkcij $f$ in $g$. Če je $\langle f, g \rangle = 0$, rečemo, da sta $f$ in $g$ \textit{ortogonalni}. Izraz $\| f \| = \sqrt{\langle f, f \rangle}$ imenujemo \textit{norma} funkcije $f$.

\end{definicija}
\vspace{0.5cm}

\begin{dogovor}

Za \textit{odsekoma zvezno} funkcijo $f$ v točkah nezveznosti vrednost funkcije v točki določimo kot
$$f(x) = \frac{1}{2} \left( \lim_{t \nearrow x} f(t) + \lim_{t \searrow x} f(t) \right).$$

\end{dogovor}
\vspace{0.5cm}

\begin{definicija}

Funkcija $f:[a, b] \rightarrow \mathbb{R}$ je \textit{odsekoma zvezno odvedljiva}, če je \textit{odsekoma zvezna} na $[a, b]$ in \text{zvezno odvedljiva} na $[a, b]$ razen morda v končno mnogo točkah, v katerih obstajata leva in desna limita odvoda.

\end{definicija}
\vspace{0.5cm}

\begin{opomba}

\textit{Odsekoma zvezno odvedljiva} funkcija ni odvedljiva v skokih in v točkah, kjer se graf $f$ zlomi, torej ima različni levo in desno tangento v tej točki. V takšni točki namreč res obstajata levi in desni odvod, saj na primer 
$$f'(x^-) = \lim_{t \nearrow x} \frac{f(t) - f(x)}{t-x} = \lim_{t \nearrow x} f'(t),$$
torej levi odvod obstaja, saj pbstaja leva limita odvodov; analogno velja za desni odvod.

Če torej v skoku spremenimo vrednost funkcije $f$ tako, da je enaka levi (desni) limiti $f$ v tej točki, potem obstaja levi (desni) odvod funckije $f$ v tej točki.

\end{opomba}
\vspace{0.5cm}

\begin{definicija}

Ker lahko vsako \textit{odsekoma zvezno odvedljivo} funkcijo na $[-\pi, \pi]$ razvijemo v Fourierovo vrsto, ki konvergira proti $f$ (vsaj po točkah), rečemo, da funkcije 
$$ \{ 1, \cos(nx), \sin(nx) \mid n \in \mathbb{N} \}$$
sestavljajo \textit{poln sistem funkcij} na intervalu $[-\pi, \pi]$.

\end{definicija}
\vspace{0.5cm}

\begin{posledica}[Fourierova sinusna in kosinusna vrsta]

Naj bo $f:[0, \pi] \rightarrow \mathbb{R}$ \textit{odsekoma zvezno odvedljiva}. Potem lahko $f$ na intervalu $[0, \pi]$ ravzijemo v \textit{sinusno} Fourierovo vrsto
$$f(x) = \sum_{n=1}^{\infty} b_n \sin(nx), ~~b_n = \frac{2}{\pi} \int_{0}^{\pi} f(x) \sin(nx) dx$$
in v \textit{kosinusno} Fourierovo vrsto
$$f(x) = a_0 + \sum_{n=1}^{\infty} a_n \cos(nx), ~~a_0 = \frac{1}{\pi} \int_{0}^{\pi} f(x) dx, ~~a_n = \frac{2}{\pi} \int_{0}^{\pi} f(x) \cos(nx) dx.$$

\end{posledica}
\vspace{0.5cm}

\begin{posledica}[Kompleksna Fourierova vrsta]

Naj bo $f:[-\pi, \pi] \rightarrow \mathbb{R}$ \textit{odsekoma zvezno odvedljiva}. Potem lahko $f$ razvijemo v \textit{kompleksno} Fourierovo vrsto
$$f(x) = \sum_{n=-\infty}^{\infty} A_n e^{inx},$$
kjer je $i = \sqrt{-1}$ \textit{imaginarna enota}, koeficienti $A_n$ za $n \in \mathbb{Z}$ pa so dani z
$$A_n = \frac{1}{2\pi} \int_{-\pi}^{\pi} f(x) e^{-inx} dx.$$

\end{posledica}
\vspace{0.5cm}

\begin{posledica}[Razvoj v Fourierovo vrsto na drugih intervalih]

Naj bo $f:[a, b] \rightarrow \mathbb{R}$ \textit{odsekoma zvezno odvedljiva}. Potem $f$ lahko razvijemo v Fourierovo vrsto po funkcijah
$$\{ 1, \cos(\frac{2n\pi}{b-a}x), \sin(\frac{2n\pi}{b-a}x) \mid n \in \mathbb{N} \}.$$
Tako dobimo
$$f(x) = a_0 + \sum_{n=1}^{\infty} \left( a_n \cos(\frac{2n\pi}{b-a}x) + b_n \sin(\frac{2n\pi}{b-a}x) \right),$$
kjer je 
$$a_0 = \frac{1}{b-a} \int_{a}^{b} f(x) dx,$$
$$a_n = \frac{2}{b-a} \int_{a}^{b} f(x) \cos(\frac{2n\pi}{b-a}x) dx,$$
$$b_n = \frac{2}{b-a} \int_{a}^{b} f(x) \sin(\frac{2n\pi}{b-a}x) dx.$$

\end{posledica}
\vspace{0.5cm}

% *************************************************************************************************

\pagebreak

% #################################################################################################

\section{NAVADNE DIFERENCIALNE ENAČBE}
\vspace{0.5cm}

% *************************************************************************************************

\subsection{Diferencialna enačba prvega reda}
\vspace{0.5cm}

\begin{definicija}

Radi bi poiskali \textit{enkrat odvedljivo} funkcijo, ki zadošča enačbi
$$y' = f(x, y).$$
Vsako tako funkcijo imenujemo \textit{rešitev} dane diferencialne enačbe, njen graf $y = y(x)$ pa \textit{rešitvena krivulja}.

Pravimo, da je z enačbo podano \textit{polje smeri}, v vsaki točki je predpisana smer, v kateri mora potekati rešitvena krivulja. To polje smeri grafično predstavimo kot družino krivulj \--- \textit{izoklin} \--- vzdolž katerih je smer \textit{konstantna}.

\end{definicija}
\vspace{0.5cm}

\begin{definicija}[Ortogonalne trajektorije]

\textit{Ortogonalne trajektorije} dane družine krivulj so take krivulje, ki v vsaki svoji točki sekajo tisto od krivulj dane družine, ki poteka skozi točko, pod pravim kotom.

Tej ortogonalni družini pripada diferencialna enačba, ki je v preprosti zvezi z diferencialno enačbo prvotne družine krivulj. V enačbi prvotne družine le zamenjamo $y'$ z $\cfrac{1}{y'}$ (kar določa pravokotno smer).

\end{definicija}
\vspace{0.5cm}

% *************************************************************************************************

\subsection{Linearna diferencialna enačba prvega reda}
\vspace{0.5cm}

% *************************************************************************************************

\subsection{Diferencialna enačba drugega reda}
\vspace{0.5cm}

% *************************************************************************************************

\pagebreak

% #################################################################################################

\end{document}