\documentclass[11pt]{article}
\usepackage[utf8]{inputenc}
\usepackage[slovene]{babel}

\usepackage{amsthm}
\usepackage{amsmath, amssymb, amsfonts}
\usepackage{relsize}
\usepackage{simpsons}
%\usepackage{geometry}
% \geometry{
% a4paper,
% total={170mm,257mm},
% left=20mm,
% top=20mm,
% }

\DeclareMathOperator{\tg}{tg}
\newcommand{\R}{\mathbb{R}}
\newcommand{\N}{\mathbb{N}}
\newcommand{\Z}{\mathbb{Z}}

\theoremstyle{definition}
\newtheorem{definicija}{Definicija}[section]

\theoremstyle{definition}
\newtheorem{trditev}{Trditev}[section]

\theoremstyle{definition}
\newtheorem{izrek}{Izrek}[section]

\theoremstyle{theorem}
\newtheorem{uporaba}{Uporaba}[section]

\newtheorem{lema}{Lema}
\newtheorem*{posledica}{Posledica}
\newtheorem*{dogovor}{Dogovor}
\newtheorem*{opomba}{Opomba}

\title{\Left\Homer ~Analiza 2 - definicije, trditve in izreki \Bart}
\author{Oskar Vavtar \\
po predavanjih profesorja Drnovška in skripti profesorja Strleta}
\date{2020/21}

\begin{document}
\maketitle
\pagebreak
\tableofcontents
\pagebreak

% #################################################################################################

\section{NEDOLOČENI INTEGRAL IN POJEM \\ DIFERENCIALNE ENAČBE}
\vspace{0.5cm}

% *************************************************************************************************

\subsection{Primitivna funkcija in nedoločeni integral}
\vspace{0.5cm}

\begin{definicija}[Primitivna funkcija]

Naj bo $f$ funkcija \textit{ene spremenljivke}. Če obstaja \textit{odvedljiva funkcija} $F: A \rightarrow \mathbb{R}$, za katero velja $F' = f$, imenujemo F \textit{primitivna funkcija} funkcije $f$ na $A$. 

\end{definicija}
\vspace{0.5cm}

\begin{lema}

Naj bosta $F$ in $G$ \textit{primitivni funkciji} za funkcijo $f$ na nekem intervalu $J$. Potem obstaja konstanta $C \in \mathbb{R}$, da velja 
$$G(x) ~=~ F(x) + C ~~~ \forall x \in J.$$

\end{lema}
\vspace{0.5cm}

\begin{definicija}[Nedoločeni integral]

\textit{Nedoločeni integral} funkcije $f$ je \hbox{skupek} vseh njenih primitivnih funkcij. Označimo ga z $\mathlarger{\int f(x)\,dx}$, funkcijo $f$ pa imenujemo \textit{integrand}.

\end{definicija}
\vspace{0.5cm}

\begin{posledica}

Naj bo $F$ neka \textit{primitivna} funkcija za $f$ na intervalu $J$. Potem je za $x \in J$
$$\int f(x)\,dx ~=~ F(x) + C,$$
kjer je $C \in \mathbb{R}$ poljubna konstanta, ki jo imenujemo \textit{splošna} ali \textit{integracijska konstanta}.

\end{posledica}
\vspace{0.5cm}

\begin{trditev}[Lastnosti nedoločenega integrala]

Za poljubni funkciji \hbox{$f$ in $g$,} ki imata primitivni funkciji na intervalu $J$, ter skalar $a \in \mathbb{R}$ velja
\begin{align*}
\int \left( f(x) \pm g(x) \right)\,dx ~&=~ \int f(x)\,dx ~\pm~ \int g(x)\,dx \\
\int a f(x)\,dx ~&=~ a \int f(x)\,dx
\end{align*}
za $x \in J$; torej je nedoločeni integral \textit{linearen}. Če je $F$ odvedljiva na $J$, potem za $x \in J$ velja
$$\int F'(x)\,dx ~=~ F(x) + C,$$
kjer je $C \in \mathbb{R}$ poljubna konstanta.

\end{trditev}
\vspace{0.5cm}

% *************************************************************************************************

\subsection{Uvedba nove spremenljivke v nedoločeni integral}
\vspace{0.5cm}

\begin{trditev}

Naj bo funkcija $g$ \textit{odvedljiva} na intervalu $J$ in naj ima funkcija $f$ primitivno funkcijo $F$ na intervalu $g(J) = \{ g(x); ~x \in J \}$. Potem je $F \circ g$ \textit{primitivna} funkcija za $(f \circ g) \cdot g'$ na $J$, torej je
$$\int f \left( g(x) \right) g'\,dx ~=~ \int f(t)\,dt,$$
kjer smo s $t = g(x)$ označili novo spremenljivko.

\end{trditev}
\vspace{0.5cm}

% *************************************************************************************************

\subsection{Integracija po delih v nedoločenem integralu}
\vspace{0.5cm}

\begin{trditev}

Naj bosta $f$ in $g$ \textit{odvedljivi} funkciji na intervalu $J$. Potem velja
$$\int f(x) g'(x)\,dx ~=~ f(x) g(x) ~-~ \int g(x) f'(x)\,dx.$$
Če označimo $u = f(x)$ in $v = g(x)$, lahko zgornjo formulo krajše zapišemo kot
$$\int u\,dv ~=~ uv ~-~ \int v\,du.$$

\end{trditev}
\vspace{0.5cm}

\pagebreak

% *************************************************************************************************

\subsection{Diferencialne enačbe 1. reda}
\vspace{0.5cm}

\begin{definicija}

\textit{Navadna diferencialna enačba} 1. reda je enačba za neznano funkcijo 
$$y~=~g(x),$$
ki vsebuje tudi odvod $y'$ funkcije $y$. \\

\noindent \textit{Splošna oblika} diferencialne enačbe 1. reda je
$$F(x,y,y')~=~0,$$
kjer je $F$ funkcija treh spremenljivk, ki je res odvisna od zadnje spremenljivke. \\

\noindent Če iz diferencialne enačbe izrazimo odvod funkcije, dobimo \textit{standardno obliko} diferencialne enačbe 1. reda
$$y' ~=~ f(x,y).$$

\noindent Začetni pogoj za diferencialno enačbo 1. reda v točki $x = a$ je podan z vrednostjo iskane funkcije $y$ v točki $a$, torej $y(a) = b$, kjer je $b \in \mathbb{R}$. Diferencialna enačba in začetni pogoj skupaj sestavljata začetno nalogo.

\end{definicija}
\vspace{0.5cm}

\begin{definicija}

\textit{Eksplicitna rešitev} diferencialne enačbe $F(x,y,y') = 0$ je takšna odvedljiva funkcija $g:J \rightarrow R$, definirana na intervalu $J$, da postane diferencialna enačba identiteta na $J$, le vanjo vstavimo $y = g(x)$, \hbox{torej $\forall x \in J$ velja} 
$$F(x,g(x),g'(x)) ~=~ 0.$$

\noindent \textit{Implicitna rešitev} diferencialne enačbe je dana z enačbo $G(x,y) = 0$, ki na nekem intervalu določa eksplicitno rešitev dane diferencialne enačbe. \\

\noindent \textit{Splošna rešitev} diferencialne enačbe 1. reda je funkcija 
$$y~=~g(x,C)~\footnote{\text{Lahko podana implicitno.}},$$ ki je odvisna od splošne konstante $C$ in reši dano diferencialno enačbo za poljubno izbiro vrednosti konstante $C \in \mathbb{R}$, poleg tega pa za poljuben začetni pogoj obstaja vrednost konstante $C$, pri kateri rešitev zadošča izbranemu začetnemu pogoju. Rešitev, ki ne vsebuje splošnih konstant, imenujemo tudi \textit{posebna} ali \textit{partikularna} rešitev.

\end{definicija}
\vspace{0.5cm}

\begin{definicija}[LDE 1. reda]

\textit{Linearna diferencialna enačba} 1. reda ima obliko
$$r_1(x)y' ~+~ r_0(x)y ~=~ s(x),$$
kjer so $r_0, r_1, s: J \rightarrow \mathbb{R}$ funkcije, definirane na nekem intervalu $J$. Če je $s$ ničelna funkcija, rečemo, da je enačba \textit{homogena}. Če sta funkciji $r_0$, $r_1$ \textit{konstantni}, pa rečemo, da ima enačba \textit{konstante koeficiente}. \\

\noindent \textit{Standardna oblika} linearne diferencialne enačbe 1. reda je
$$y' ~+~ p(x)y ~=~ q(x),$$
kjer sta $p, q: J \rightarrow \mathbb{R}$ funkciji, definirani na intervalu $J$.

\end{definicija}
\vspace{0.5cm}

\begin{trditev}

Naj bo dana \textit{linearna} diferencialna enačba 
$$y' ~+~ p(x) y ~=~ q(x),$$
kjer sta $p, q: J \rightarrow \mathbb{R}$ \textit{zvezni} funkciji, definirani na intervalu $J$.
\begin{enumerate}

	\item[(i)] Splošna rešitev pripadajoče homogene diferencialne enačbe \\ $y' + p(x) y = 0$ je
	$$y(x) ~=~ Ce^{-\int p(x)\,dx},$$
	kjer je $C \in \mathbb{R}$ splošna konstanta, v kateri je zajeta integracijska konstanta.
	
	\item[(ii)] Splošna rešitev dane diferencialne enačbe dobimo iz splošne rešitve pripadajoče homogene enačbe z variacijo konstante, torej z nastavkom
	$$y(x) ~=~ C(x)e^{-\int p(x)\,dx},$$
	kjer funkcija $C(x)$ zadošča
	$$C'(x) ~=~ q(x)e^{\int p(x)\,dx}.$$

\end{enumerate}

\end{trditev}
\vspace{0.5cm}

% *************************************************************************************************

\pagebreak

% #################################################################################################

\section{DOLOČENI INTEGRAL}
\vspace{0.5cm}

% *************************************************************************************************

\subsection{Motivacija za določeni integral}
\vspace{0.5cm}

\begin{definicija}

Naj bo $f:[a, b] \rightarrow \mathbb{R}$ \textit{nenegativna funkcija}, torej \hbox{$f(x) \geq 0$} $\forall x \in [a, b]$. Rečemo, da graf funkicje $f$ \textit{določa območje} $A \subset \mathbb{R}^2$  nad \hbox{intervalom} $[a, b]$. Množica $A$ je navzgor omejena z grafom funkcije $f$, na levi s premico $x=a$ in na desni s premico $x=b$.

\end{definicija}
\vspace{0.5cm}

% *************************************************************************************************

\subsection{Riemannova vsota in Riemannov integral}
\vspace{0.5cm}

\begin{definicija}[Riemannova vsota]

\textit{Delitev} $D$ intervala $[a, b]$ na \hbox{podintervale} je dana z izbiro \textit{delilnih točk} $x_i$:
$$a ~=~ x_0 < x_1 < x_2 < \ldots < x_{n-1} < x_n ~=~ b,$$
kjer je $n \in \mathbb{N}$. Dolžino $i$-tega podintervala $[x_{i-1},x_i]$ (za $i=1,2,...,n$) označimo z $\delta_i := x_i - x_{i-1}$. \textit{Velikost delitve} $D$ je dolžina najdaljšega \hbox{podintervala} delitve $D$, torej
$$\delta(D) ~=~ \max{\{\delta_i \mid i = 1, 2, ..., n\}}.$$

\noindent Na vsakem od podintervalov, na katere delitev $D$ razdeli interval $[a, b]$, \hbox{izberemo} \textit{testno točko} $t_i \in [x_{i-1}, x_i]$ in s $T_D = (t_1, t_2, \dots, t_n)$ označimo nabor teh točk; nabor testnih točk je \textit{usklajen} z delitvijo $D$, ker smo na vsakem podintervalu $[x_{i-1},x_i]$, določenem z $D$, izbrali natanko eno testno točko $t_i$. \\

\noindent \textit{Riemannova vsota} funkcije $f:[a, b] \rightarrow \mathbb{R}$, pridružena delitvi $D$ in \hbox{usklajenemu} naboru testnih točk $T_D$ je 
$$R(f, D, T_D) ~:=~ \sum_{i=1}^{n} f(t_i) \delta_i.$$

\end{definicija}
\vspace{0.5cm}

\pagebreak

\begin{definicija}[Riemannov integral]

\textit{Riemannov integral} ali \hbox{\textit{določeni integral}} funkcije $f: [a, b] \rightarrow \mathbb{R}$ je limita Riemannovih vsot $R(f, D, T_D)$, kjer limito vzamemo po vseh delitvah D intervala [a, b] in usklajenih naborih testnih točk $T_D$, ko pošljemo velikost delitev $\delta(D)$ proti $0$, če ta limita obstaja (torej je končna in neodvisna od izbire delitev in testnih točk). Pišemo
$$\int_{a}^{b} f(x)\,dx ~:=~ \lim_{\delta(D) \rightarrow 0} R(f, D, T_D).$$
Če zgornja limita obstaja, rečemo, da je funkcija $f$ \textit{integrabilna} na $[a, b]$.

\end{definicija}
\vspace{0.5cm}

\begin{definicija}

$$\lim_{\delta(D) \rightarrow 0} R(f, D, T_D) ~=~ I,$$
če $\forall \varepsilon > 0 ~ \exists \delta > 0$, da za poljubno delitev $D$ z $\delta(D) < \delta$ in poljuben usklajen nabor testnih točk $T_D$ velja
$$|R(f, D, T_D) - I| ~<~ \varepsilon.$$ 

\end{definicija}
\vspace{0.5cm}

\begin{trditev}[Linearnost določenega integrala]

Naj bosta $f, g: [a, b] \rightarrow \mathbb{R}$ \textit{integrabilni funkciji} in $c \in \mathbb{R}$. Potem so $f \pm g$ in $cf$ \textit{integrabilne} na $[a, b]$ in velja
\begin{align*}
\int_a^b \left( f(x) \pm g(x) \right)\,dx ~&=~ \int_a^b f(x)\,dx ~\pm~ \int_a^b g(x)\,dx \\
\int_a^b cf(x)\,dx ~&=~ c\int_a^b f(x)\,dx 
\end{align*}

\end{trditev}
\vspace{0.5cm}

% *************************************************************************************************

\subsection{Integrabilne funkcije}
\vspace{0.5cm}

\begin{trditev}

Naj bo $f$ \textit{integrabilna} na $[a, b]$. Potem je $f$ omejena na $[a, b]$.

\end{trditev}
\vspace{0.5cm}

\begin{definicija}[Zožitev]

Naj bo $f:A \rightarrow \mathbb{R}$ funkcija in $B \subset A$. Tedaj $f |_B: B \rightarrow \mathbb{R}$ označuje funkcijo z definicijskim območjem $B$, ki $\forall x \in B$ preslika v $f(x)$. Funkcijo $f |_B$ imenujemo \textit{zožitev} funkcije $f$ na $B$.   

\end{definicija}
\vspace{0.5cm}

\begin{trditev}[Aditivnost domene]

Naj bodo $a < b < c$ in $f:[a, c] \rightarrow \mathbb{R}$. Tedaj je $f$ \textit{integrabilna} na $[a, c]$ natanko tedaj, ko sta \textit{integrabilni} zožitvi $f |_{[a, b]}$ in $f |_{[b, c]}$. V tem primeru velja
$$\int_a^c f(x)\,dx ~=~ \int_a^b f(x)\,dx ~+~ \int_b^c f(x)\,dx.$$

\end{trditev}
\vspace{0.5cm}

\begin{trditev}

Naj bosta  $f, g: [a, b] \rightarrow \mathbb{R}$ funkciji, ki se razlikujeta le v točki $c \in [a, b]$ \footnote{$f(x) = g(x)$ $\forall x \in [a, b]$, $x \neq c$}. Potem je $f$ \textit{integrabilna} natanko tedaj, ko je \textit{integrabilna} $g$; če sta funkciji \textit{integrabilni} velja
$$\int_a^b f(x)\,dx ~=~ \int_a^b g(x)\,dx.$$
 
\end{trditev}
\vspace{0.5cm}

\begin{izrek}

Naj bo $f$ \textit{zvezna} na $[a, b]$. Potem je $f$ \textit{integrabilna} na $[a, b]$.

\end{izrek}
\vspace{0.5cm}

\begin{definicija}[Enakomerna zveznost]

Naj bo $A \subseteq \mathbb{R}^n$. Funkcija \hbox{$f: A \rightarrow \mathbb{R}$} je \textit{enakomerno zvezna} na $A$, če $\forall \varepsilon > 0$ obstaja tak $\delta = \delta_\varepsilon > 0$, da za poljubna $x, y \in A$, ki zadoščata $|x-y| < \delta$, velja
$$|f(x) - f(y)| ~<~ \varepsilon.$$ 
Tukaj je za $x = (x_1, \ldots, x_n) \in \mathbb{R}^n$ dolžina $|x|$ definirana z
$$|x| ~=~ \sqrt{\sum_{k=1}^n {x_k}^2} ~=~ \sqrt{{x_1}^2 + \ldots + {x_n}^2}.$$

\end{definicija}
\vspace{0.5cm}

\begin{trditev}

Naj bo $A \subseteq \mathbb{R}^n$ \textit{kompaktna množica} in $f: A \rightarrow \mathbb{R}$ \textit{zvezna} funkcija. Tedaj je $f$ \textit{enakomerno zvezna} na $A$.

\end{trditev}
\vspace{0.5cm}

\begin{definicija}[Odsekoma zvezna funkcija]

Funkcija $f: J \rightarrow \mathbb{R}$, definirana na omejenem intervalu $J$, je \textit{odsekoma zvezna}, če je zvezna v vseh točkah intervala razen morda v končno mnogo točkah, kjer ima skoke. \\

\noindent Funkcija f ima \textit{skok} v točki $c \in J$, če $f$ ni zvezna v $c$, ima pa (končno) levo in desno limito $c$ (če je $c$ krajišče intervala, zahtevamo le obstoj limite na tisti strani $c$, ki leži v $J$). 

\end{definicija}
\vspace{0.5cm}

\begin{posledica}

Če je $f:[a, b] \rightarrow \mathbb{R}$ \textit{odsekoma zvezna}, potem je \textit{integrabilna}. Vrednosti funkcije f v skokih ne vplivajo niti na integrabilnost niti na integral funkcije $f$ na $[a, b]$.

\end{posledica}
\vspace{0.5cm}

\begin{trditev}
~
\begin{enumerate}
	
	\item[(i)] Naj bosta $f$ in $g$ \textit{integrabilni} na $[a, b]$. Če je $f(x) \leq g(x)$ $\forall x \in [a, b]$, potem je 
	$$\int_a^b f(x)\,dx ~\leq~ \int_a^b g(x)\,dx ~~~\text{(monotonost integrala)}.$$
	
	\item[(ii)] Če je $f$ \textit{integrabilna} na $[a, b]$, potem je tudi $|f|$ \textit{integrabilna} na $[a, b]$ in velja
	$$\left| \int_a^b f(x)\,dx \right| ~\leq~ \int_a^b |f(x)|\,dx.$$	
	
\end{enumerate}
\end{trditev}
\vspace{0.5cm}

\begin{dogovor}
~
	\begin{itemize}
		\item Integral po izrojenemu intervalu $[a, a]$ je nič: 
		$$\int_{a}^{a} f(x)\,dx ~=~ 0.$$ \\
		\item Če je $a < b$, je 
		$$\int_{b}^{a} f(x)\,dx ~=~ -\int_{a}^{b} f(x)\,dx.$$
	\end{itemize}
\end{dogovor}
\vspace{0.5cm}

\begin{definicija}[Povprečna vrednost]

\textit{Povprečna vrednost} integrabilne \hbox{funkcije $f$} na intervalu $[a, b]$ je 
$$\mu ~:=~ \frac{1}{b-a} \int_{a}^{b} f(x)\,dx.$$

\end{definicija}
\vspace{0.5cm}

\begin{trditev}

Naj bo $f:[a, b] \rightarrow \mathbb{R}$ \textit{integrabilna}, $m := \inf{f}$ in $M := \sup{f}$. Potem je za povprečno vrednost $\mu$ funkcije $f$ velja $m \leq \mu \leq M$. Če je $f$ \textit{zvezna}, obstaja taka točka $c \in [a, b]$, da je $\mu = f(c)$.

\end{trditev}
\vspace{0.5cm}

% *************************************************************************************************

\subsection{Osnovni izrek analize}
\vspace{0.5cm}

\begin{definicija}

Naj bo $f:[a, b] \rightarrow \mathbb{R}$ \textit{integrabilna} funkcija. Funkcijo \\$F:[a, b] \rightarrow \mathbb{R}$, definirano s predpisom
$$F(x) ~=~ \int_{a}^{x} f(t)\,dt,$$
imenujemo \textit{integral kot funkcija zgornje meje}.

\end{definicija}
\vspace{0.5cm}

\begin{izrek}[Prvi del osnovnega izreka]

Naj bo $f:[a, b] \rightarrow \mathbb{R}$ \textit{zvezna}. Potem je funkcija $F(x) = \mathlarger{\int_a^x f(t)~dt}$ \textit{odvedljiva} na $[a, b]$ in velja
$$F'(x) ~=~ f(x) ~~~ \forall x \in [a, b].$$

\end{izrek}
\vspace{0.5cm}

\begin{posledica}

Naj bo $f:[a, b] \rightarrow \mathbb{R}$ \textit{zvezna}. Potem je $F(x) = \mathlarger{\int_a^x f(t)~dt}$ \textit{primitivna funkcija} za $f$ na $[a, b]$. Torej velja
$$\int f(x)\,dx ~=~ \int_a^x f(t)\,dt ~+~ C,$$
kjer je $C \in \mathbb{R}$ poljubna konstanta.

\end{posledica}
\vspace{0.5cm}

\begin{izrek}[Drugi del osnovnega izreka]

Naj bo $f:[a, b] \rightarrow \mathbb{R}$ \textit{zvezna} in $G$ poljubna primitivna funkcija za $f$ na $[a, b]$. Potem je
$$\int_a^b f(x)\,dx ~=~ G(b) - G(a) ~=~ G(x) \Big|_a^b.$$

\end{izrek}
\vspace{0.5cm}

% *************************************************************************************************

\subsection{Pravila za integriranje in Leibnizova formula}
\vspace{0.5cm}

\begin{trditev}
~
\begin{enumerate}

	\item[(i)] Naj bo $g:[a, b] \rightarrow \mathbb{R}$ \textit{zvezno odvedljiva} in $f: Z_g \rightarrow \mathbb{R}$ \textit{zvezna}. Potem ob uvedbi nove spremenljivke $t = g(x)$ velja:
	$$\int_a^b f(g(x))\,g'(x)\,dx ~=~ \int_{g(a)}^{g(b)} f(t)\,dt.$$
	
	\item[(ii)] Naj bosta $f,g:[a, b] \rightarrow \mathbb{R}$ \textit{zvezno odvedljivi}. Potem je
	$$\int_a^b f(x)\,g'(x)\,dx ~=~ f(x)\,g(x) \Big|_a^b ~-~ \int_a^b g(x)\,f'(x)\,dx.$$
	Če označimo $u = f(x)$ in $v = g(x)$, zgornja formula postane
	$$\int_a^b u\,dv ~=~ uv \Big|_a^b ~-~ \int_a^b v\,du.$$
	
\end{enumerate}
\end{trditev}
\vspace{0.5cm}

% *************************************************************************************************

\subsection{Posplošeni integral na omejenem intervalu}
\vspace{0.5cm}

\begin{definicija}[Posplošeni integral]

Naj bo $f:(a, b] \rightarrow \mathbb{R}$ funkcija, ki je integrabilna na intervalu $[t, b]$ $\forall t \in (a, b)$. Potem je \textit{posplošeni integral} funkcije $f$ na intervalu $[a, b]$
$$\int_{a}^{b} f(x)\,dx ~:=~ \lim_{t \searrow a} \int_{t}^{b} f(x)\,dx,$$
če ta limita obstaja. \\

\noindent Če limita obstaja , rečemo, da je $f$ \textit{posplošeno integrabilna} na $[a, b]$ in da je $\mathlarger{\int_{a}^{b} f(x)\,dx}$ \textit{konvergenten}, sicer pa rečemo, da je integral \textit{divergenten}.

\end{definicija}
\vspace{0.5cm}

\begin{opomba}
~
\begin{itemize}
	
	\item Posplošeni integral imenujemo tudi \textit{izlimitirani integral} ali \hbox{\textit{nepravi integral}}.
	
	\item Če je $f$ \textit{integrabilna} na $[a, b]$, potem je njen posplošeni integral na $[a, b]$ enak Riemannovemu integralu.	
	
\end{itemize}
\end{opomba}
\vspace{0.5cm}

\begin{trditev}

Naj bo $p \in \mathbb{R}$. Posplošeni integral $\mathlarger{\int_a^b \frac{1}{(x-a)^p}\,dx}$ je \hbox{\textit{konvergenten}} natanko tedaj, ko je $p<1$.

\end{trditev}
\vspace{0.5cm}

\begin{izrek}[Konvergenčni kriterij]

Naj bo $g:(a, b] \rightarrow \mathbb{R}$ \textit{zvezna}.
\begin{enumerate}
		
	\item[(i)] Če je $g$ \textit{omejena} na $(a, b]$ in je $p<1$, potem je $\mathlarger{\int_a^b \frac{g(x)}{(x-a)^p}\,dx}$ \hbox{\textit{konvergenten}}.
	
	\item[(ii)] Če je $g$ \textit{omejena stran od nič}, torej obstaja neki $m > 0$, da velja $|g(x)| \geq m$ $\forall x \in (a, b]$ in je $p \geq 1$, potem je $\mathlarger{\int_a^b \frac{g(x)}{(x-a)^p}\,dx}$ \textit{divergenten}.		
		
\end{enumerate}
\end{izrek}
\vspace{0.5cm}

\begin{opomba}
~
\begin{itemize}

	\item Funkcija $g$ v izreku je lahko le \textit{odsekoma zvezna} na $(a, b]$, saj je takšna $g$ \textit{zvezna} na nekem manjšem intervalu $(a, c]$. Ker je integrand potem \textit{odsekoma zvezna} funkcija na $[c, b]$, moramo obravnavati le konvergenco na $(a, c]$.
	
	\item Podobno je v točki (ii) dovolj, da je $g$ \textit{omejena stran od nič} le na manjšem intervalu $(a, c] \subset (a, b]$.
	
	\item Pogoj v točki (i) je izpoljen, če ima $g$ (končno) desno limito pri $a$. Podobno je pogoj v (ii) izpoljen, le ima $g$ od nič različno desno limito v $a$.

\end{itemize}
\end{opomba}
\vspace{0.5cm}

\begin{lema}

Naj bo $f:(a, b) \rightarrow \mathbb{R}$ \textit{monotona} in \textit{omejena} funkcija. Potem obstajata limiti
$$\lim_{x \nearrow b} f(x) ~~~\text{in}~~~ \lim_{x \searrow a} f(x);$$
ena od teh limit je enaka $\sup{f}$, druga pa $\inf{f}$.

\end{lema}
\vspace{0.5cm}

\begin{trditev}[Linearnost posplošenega integrala]

Naj bosta \hbox{$f,g: (a, b] \rightarrow \mathbb{R}$} \textit{posplošeno integrabilni} in $c \in \mathbb{R}$. Potem so tudi $f \pm g:(a, b] \rightarrow \mathbb{R}$ in $cf:(a, b] \rightarrow \mathbb{R}$ \textit{posplošeno integrabilne} in velja 
\begin{align*}
\int_a^b \left( f(x) \pm g(x) \right)\,dx ~&=~ \int_a^b f(x)\,dx ~\pm~ \int_a^b g(x)\,dx \\
\int_a^b c f(x)\,dx ~&=~ c \int_a^b f(x)\,dx.
\end{align*}

\end{trditev}
\vspace{0.5cm}

\begin{opomba}

Pri oznakah v zgornji trditvi velja naslednje: če sta dve od funkcij $f$, $g$ in $f \pm g$ posplošeno integrabilni na $[a, b]$, potem je tudi tretja. En primer je zajet v trditvi, iz integrabilnosti $f$ in $f+g$ pa sledi integrabilnost $g$, saj jo lahko zapišemo kot linearno kombinacijo drugih dveh: $g ~=~ (f+g) - f$.

\end{opomba}
\vspace{0.5cm}

\begin{definicija}
~
\begin{enumerate}

	\item[(i)] Naj bo $f:[a, b) \rightarrow \mathbb{R}$ \textit{integrabilna} na $[a, b]$ $\forall s \in (a, b)$. Potem je \textit{posplošeni integral} funkcije $f$ na intervalu $[a, b]$
	$$\int_a^b f(x)\,dx ~:=~ \lim_{s \nearrow b} \int_a^s f(x)\,dx,$$
	če ta limita obstaja. Če limita obstaja, rečemo, da je $f$ \textit{posplošeno integrabilna} na $[a, b]$ in da je posplošeni integral \textit{konvergenten}, sicer pa imenujemo integral \textit{divergenten}.
	
	\item[(ii)]	Naj bo $f:[a, c) \cup (c, b] \rightarrow \mathbb{R}$ \textit{integrabilna} na $[a, s]$ $\forall s \in (a, c)$ in \textit{integrabilna} na $[a, b]$ $\forall t \in (c, b)$. Potem je \textit{posplošeni integral} $f$ na $[a, b]$
	$$\int_a^b f(x)\,dx ~:=~ \lim_{s \nearrow c} \int_a^s f(x)\,dx ~+~ \lim_{t \searrow c} \int_t^b f(x)\,dx,$$
 če obe limiti obstajata.	
 
\end{enumerate}
\end{definicija}
\vspace{0.5cm}

% *************************************************************************************************

\subsection{Posplošeni integral na neomejenem intervalu}
\vspace{0.5cm}

\begin{definicija}[Posplošena integrabilnost]
~
	\begin{itemize}
	
		\item Naj bo $f:[a, \infty) \rightarrow \mathbb{R}$ integrabilna na $[a, s]$ $\forall s > a$. Potem je \textit{\hbox{posplošeni} integral} funkcije $f$ na $[a, \infty)$
		$$\int_{a}^{\infty} f(x)\,dx ~:=~ \lim_{s \rightarrow \infty} \int_{a}^{s} f(x)\,dx,$$
		če ta limita obstaja. Če limita obstaja, rečemo, da je posplošeni \hbox{integral} \textit{konvergenten}, sicer pa, da je \textit{divergenten}. \\
		
		\item Naj bo $f:(-\infty, b] \rightarrow \mathbb{R}$ integrabilna na $[t, b]$ $\forall t < b$. Potem je \textit{posplošeni integral} funkcije $f$ na $(-\infty, b]$
		$$\int_{-\infty}^{b} f(x)\,dx ~:=~ \lim_{t \rightarrow -\infty} \int_{t}^{b} f(x)\,dx,$$
		če ta limita obstaja. Če limita obstaja, rečemo, da je posplošeni \hbox{integral} \textit{konvergenten}, sicer pa, da je \textit{divergenten}. \\
		
		\item Funkcija $f:(-\infty, \infty) \rightarrow \mathbb{R}$ je \textit{posplošeno integrabilna}, če sta \hbox{posplošeno} integrabilni zožitvi $f |_{(-\infty, a]}$ in $f |_{[a, \infty)}$ $\forall a \in \mathbb{R}$. 
		
	\end{itemize}
\end{definicija}
\vspace{0.5cm}

\begin{izrek}[Konvergenčni kriterij]

Naj bo $g:[a, \infty) \rightarrow \mathbb{R}$ \textit{zvezna}, $a > 0$.
\begin{enumerate}

	\item[(i)] Če je $g$ \textit{omejena} na $[a, \infty)$ in je $p>1$, potem je $\mathlarger{\int_a^{\infty} \frac{g(x)}{x^p}~dx}$ \hbox{\textit{konvergenten}}.
	
	\item[(ii)] Če je $g$ \textit{omejena stran od nič} na $[a, \infty)$ in je $p \leq 1$, potem je $\mathlarger{\int_a^{\infty} \frac{g(x)}{x^p}~dx}$ \textit{divergenten}.

\end{enumerate}

\end{izrek}
\vspace{0.5cm}

% *************************************************************************************************

\pagebreak

% #################################################################################################

\section{KRIVULJE V RAVNINI}
\vspace{0.5cm}

% *************************************************************************************************

\subsection{Podajanje krivulj}
\vspace{0.5cm}

\begin{itemize}
	\item \textsc{Eksplicitno:} Funkcija $f: J \rightarrow \mathbb{R}$ za $J \subseteq \mathbb{R}$ določa krivuljo $\Gamma_f$, ki je graf te funkcije, torej
	$$\Gamma_f ~=~ \{(x, f(x)) \mid x \in J\}.$$
	\item \textsc{Implicitno:} Funkcija $g: A \rightarrow \mathbb{R}$ za $A \subseteq \mathbb{R}^2$ določa krivuljo $K_g$, ki je množica rešitev enačbe $g(x, y) = 0$, torej
	$$K_g ~=~ \{(x, y) \in A \mid g(x, y) = 0\}.$$
	\item \textsc{Parametrično:} Funkciji $\alpha, \beta: J \rightarrow \mathbb{R}$ za $J \subseteq \mathbb{R}$ določata krivuljo $K_F$, ki je množica vseh točk $(x, y)$, določenih z $x = \alpha(t)$ in $y = \beta(t)$, torej
	$$K_F ~=~ \{(\alpha(t), \beta(t)) \mid t \in J\}.$$
	Preslikavo $F: J \rightarrow \mathbb{R}^2$, $F(t) := (\alpha(t), \beta(t))$ imenujemo \textit{pot} ali \\\textit{parametrizacija} krivulje $K_F$. Krivuljo $K_F$ imenujemo tudi \textit{tir} poti $F$.
	\item \textsc{Polarno:} Funkcija $h: J \rightarrow \mathbb{R}$ za $J \subseteq \mathbb{R}$ določa krivuljo $K_h$, ki je množica točk v ravnini s polarnima koordinatama $(r, \varphi)$, kjer je $r = h(\varphi)$, torej
	$$K_h ~=~ \{(h(\varphi)\cos{\varphi}, h(\varphi)\sin{\varphi}) \mid \varphi \in J\}.$$
	Zveza med kartezičnima koordinatama $(x, y)$ ter polarnima \hbox{koordinatama} $(r, \varphi)$ točje v ravnini:
	$$x ~=~ r \cos{\varphi}, ~~~y ~=~ r \sin{\varphi}, ~~~r ~=~ \sqrt{x^2 + y^2}, ~~~\tg{\varphi} ~=~ \frac{y}{x}.$$
\end{itemize}
\vspace{0.5cm}

% *************************************************************************************************

\subsection{Enačba tangente na krivuljo}
\vspace{0.5cm}

\begin{trditev}

Naj bo $f: J \rightarrow \R$ \textit{odvedljiva} v notranji točki $c$ intervala $J \subseteq \R$. Potem ima krivulja $\Gamma_f$ tangento v točki $(c, f(c))$, dano z enačbo:
$$y - f(c) ~=~ f'(c)(x-c).$$
Naj bo dana \textit{zvezno odvedljiva} funkcija $g: A \rightarrow \R$ in notranja točka \hbox{$(a, b) \in A \subseteq \R^2$}, ki zadošča $g(a, b) = 0.$ Po izreku o \textit{implicitni funkciji} enačba $g(x, y) = 0$ določa funkcijo $y = f(x)$ v okolici točke $a$, če je $\cfrac{\partial g}{\partial y}(a, b) \neq 0$: za to funkcijo velja $f(a) = b$ in 
$$f'(a) ~=~ -\frac{g_x(a,b)}{g_y(a, b)}.$$

\end{trditev}
\vspace{0.5cm}

\begin{definicija}[Regularna in singularna točka]

Naj bo $g: A \rightarrow \mathbb{R}$ \hbox{odvedljiva} v točki $(a, b) \in A \subseteq \mathbb{R}^2$. Če je 
$$\nabla g(a, b) ~\neq~ (0, 0),$$
rečemo, da je $(a, b)$ \textit{regularna točka} za $g$, sicer pa, da je $(a, b)$ \textit{singularna točka} za $g$. 

\end{definicija}
\vspace{0.5cm}

\begin{trditev}

Naj bo $g: A \rightarrow \R$ \textit{zvezno odvedljiva} v okolici točke \hbox{$(a, b) \in A \subseteq \R^2$}, naj bo $g(a, b) = 0$ in naj bo $(a, b)$ \textit{regularna} točka za $g$. Potem ima krivulja $K_g$ tangento v točki $(a, b)$, dano z enačbo
$$g_x(a, b)(x-a) ~+~ g_y(a, b)(y-b) ~=~ 0$$
oziroma
$$\nabla{g(a, b)} \cdot (x-a, y-b) ~=~ 0.$$

\end{trditev}
\vspace{0.5cm}

\begin{trditev}

Naj bosta $\alpha, \beta: J \rightarrow \R$ \textit{zvezno odvedljivi} funkciji, kjer je $J \subset \R$ interval, ter $F = (\alpha, \beta)$ pripadajoča \textit{zvezno odvedljiva} pot. Če velja $\dot{\alpha}(t) \neq 0$ $\forall t \in J$, potem je $K_F$ graf neke \textit{zvezno odvedljive} funkcije $f$, za katero velja
$$f'(\alpha(t)) ~=~ \frac{\dot{\beta}(t)}{\dot{\alpha}(t)}.$$
Če označimo $x(t) := \alpha(t)$ in $y(t) := \beta(t)$, dobimo za odvod
$$f'(x(t)) ~=~ \frac{\dot{y}(t)}{\dot{x}(t)}.$$

\end{trditev}
\vspace{0.5cm}

\begin{posledica}

Naj bosta $x = x(t)$ in $y = y(t)$ \textit{dvakrat odvedljivi} na intervalu $J$ in naj velja $\dot{x}(t) \neq 0$ $\forall t \in J$. Potem je pripadajoča funkcija $f$ iz zgornje trditve \textit{dvakrat odvedljiva} in velja
$$f''(x(t)) ~=~ \frac{\dot{x}(t) \ddot{y}(t) - \dot{y}(t) \ddot{x}(t)}{\dot{x}(t)^3}.$$

\end{posledica}
\vspace{0.5cm}

\begin{definicija}

Naj bosta $\alpha, \beta: J \rightarrow \mathbb{R}$ \textit{odvedljivi}, kjer je $J \subseteq \mathbb{R}$ interval, ter $F = (\alpha, \beta)$ pripadajoča \textit{odvedljiva} pot. \textit{Odvod poti} $F$ po $t$ je hitrostni vektor $\dot{F}(t) = (\dot{\alpha}(t), \dot{\beta}(t))$. 
Če je 
$$\dot{F}(t) ~\neq~ (0, 0)$$ 
za neki $t \in J$, imenujemo $t$ \textit{regularna točka} parametrizacije $F$. Če so vse točke intervala $J$ regularne, imenujemo $F$ \textit{regularna parametrizacija}. \\
Naj bo $g: I \rightarrow J$ \textit{odvedljiva surjektivna} funkcija, kjer je $I \subset \mathbb{R}$ interval. Pot
$$G ~:=~ F \circ g$$
imenujemo \textit{reparametrizacija} poti $F$.

\end{definicija}
\vspace{0.5cm}

\begin{trditev}

Naj bo $s$ \textit{regularna točka} parametrizacije $F(t) = (\alpha(t), \beta(t))$. Potem je 
tangenta na krivuljo $K_F$ v točki $F(s)$ dana z enačbo
$$\dot{\alpha}(s)(y - \beta(s)) ~=~ \dot{\beta}(s)(x - \alpha(s)).$$

\end{trditev}
\vspace{0.5cm}

% *************************************************************************************************

\subsection{Dolžina loka krivulje}
\vspace{0.5cm}

\begin{definicija}

Naj bo dana pot $F:[a, b] \rightarrow \mathbb{R}^2$, $F(t) = (\alpha(t), \beta(t))$, ki določa krivuljo $K$. Izberimo delitev
$$D ~=~ \{a = t_0 < t_1 < \ldots < t_n = b\}$$
intervala $[a, b]$. Pot $F(t)$ na $i$-tem podintervalu $[t_{i-1}, t_i]$ zamenjamo z daljico od $F(t_{i-1})$ do $F(t_i)$. \\
Dolžina tako nastale lomljene črte, ki aproksimira tir poti $F$, je 
$$\ell(D) ~=~ \sum_{i = 1}^{n} \sqrt{ (\alpha(t_i) - \alpha(t_{i-1}))^2 ~+~ (\beta(t_i) - \beta(t_{i-1}))^2 }.$$
Če obstaja limita dolžin $\ell(D)$, ko pošljemo velikost delitve $\delta(D)$ proti nič (neodvisno od izbire delitev), jo imenujemo \textit{dolžina poti} $F$ in označimo $\ell(F)$:
$$\ell(F) ~=~ \lim_{\delta(D) \rightarrow 0} \ell(D).$$

\end{definicija}
\vspace{0.5cm}

\begin{trditev}

Naj bo pot $F: [a, b] \rightarrow \R^2$, $F(t) = (x(t), y(t))$ \hbox{\textit{zvezno odvedljiva}}\footnote{Komponenti $x(t)$ in $y(t)$ sta \textit{zvezno odvedljivi} funkciji}. Potem je dolžina $\ell(F)$ poti $F$ enaka
$$\ell(F) ~=~ \int_a^b \sqrt{\dot{x}(t)^2 ~+~ \dot{y}(t)^2}\,dt.$$

\end{trditev}
\vspace{0.5cm}

\begin{opomba}
~
\begin{enumerate}
	\item Dolžina tangentnega vektorja $\dot{F}(t) ~=~ (\dot{x}(t), \dot{y}(t))$ na pot $F$ je \\$|\dot{F}(t)| = \sqrt{\dot{x}(t)^2 + \dot{y}(t)^2}$, torej je
	$$\ell(F) ~=~ \int_a^b |\dot{F}(t)|\,dt.$$
	
	\item Iz zgornje formule za dolžino poti se zdi, da je ta odvisna od parametrizacije $F$. Na osnovi definicije pa pričakujemo, da je odvisna le od krivulje $K$, ki je tir poti. To je res, če je parametrizacija \textit{injektivna}, torej vsako točko na krivulji obišče natanko enkrat; če pa parametrizacija večkrat opiše kak del krivulje, tega v dolžini upoštevamo večkrat.
\end{enumerate}
\end{opomba}
\vspace{0.5cm}

\begin{trditev}

Naj bo $F: [a, b] \rightarrow \R^2$ \textit{zvezno odvedljiva} parametrizacija krivulje $K$ in $g: [c, d] \rightarrow [a, b]$ \textit{zvezno odvedljiva} funkcija, ki je \textit{monotono naraščajoča} in velja $g(c) = a$ ter $g(d) = b$. Potem je $G := F \circ g: [c, d] \rightarrow \R^2$ tudi parametrizacija $K$ in velja 
$$\ell(G) ~=~ \ell(F).$$
Dolžino krivulje $K$ torej izračunamo kot dolžino njene poljubne injektivne parametrizacije.

\end{trditev}
\vspace{0.5cm}

\begin{posledica}
~
\begin{enumerate}
	\item[(i)] Dolžina \textit{eksplicitno} podane \textit{zvezno odvedljive} krivulje
	$y = f(x)$ za \hbox{$x \in [a, b]$} je
	$$\ell(K_f) ~=~ \int_a^b \sqrt{1 ~+~ (f'(x))^2}\,dx ~=~ \int_a^b \sqrt{1 + {y'}^2}\,dx.$$

	\item[(ii)] Dolžina \textit{polarno} podane \textit{zvezno odvedljive} krivulje $r = 	r(\varphi)$ za $\varphi \in [a, b]$ je
	$$\ell(K_r) ~=~ \int_a^b \sqrt{r(\varphi)^2 ~+~ \dot{r}(\varphi)^2}\,d\varphi.$$
\end{enumerate}
\end{posledica}
\vspace{0.5cm}

\begin{definicija}[Ločna dolžina]

Diferencial dolžina loka krivulje označimo z $ds$ in ga imenujemo \textit{ločna dolžina}. V vseh opisih krivulje velja 
$$ds^2 ~=~ dx^2 + dy^2.$$

\end{definicija}
\vspace{0.5cm}

\begin{uporaba}[Površina rotacijske ploskve]

Naj bo $f:[a, b] \rightarrow \mathbb{R}$ \\\textit{nenegativna zvezna} funkcija. Ploskev, ki jo dobimo z vrtenjem grafa funkcije $f$ nad intervalom $[a, b]$ okoli osi $x$, imenujemo \textit{rotacijska ploskev}. \\

\noindent Izberemo neko delitev
$$D ~=~ \{a = x_0 < x_1 < \ldots < x_n = b\}$$
intervala $[a, b]$. Nad intervalom $[x_{i-1}, x_i]$ graf funkcije $f$ aproksimiramo z daljico od točke $(x_{i-1}, f(x_{i-1})$ do točke $(x_i, f(x_i))$. Ko daljico zavrtimo okoli $x$-osi, dobimo plašč prisekanega stožca s polmeroma leve in desne mejne krožnice $f(x_{i-1})$ in $f(x_i)$ ter višino $\delta_i = x_i - x_{i-1}$. To da približek za površino ploskve:
$$\sum_{i=1}^{n} \pi (f(x_{i-1}) + f(x_i)) \sqrt{{\delta_i}^2 ~+~ (f(x_{i-1}) - f(x_i))^2}.$$
Če je $f$ \textit{zvezno odvedljiva}, dobimo za površino v limiti, ko pošljemo velikost delitve $\delta(D)$ proti $0$, formulo
$$P ~=~ 2 \pi \int_{a}^{b} f(x) \sqrt{1 ~+~ (f'(x))^2}\,dx ~=~ 2 \pi \int_{a}^{b} y \sqrt{1 + {y'}^2}\,dx.$$

\end{uporaba}
\vspace{0.5cm}

% *************************************************************************************************

\subsection{Ploščina območja, določenega s krivuljo}
\vspace{0.5cm}

\begin{trditev}
~
\begin{enumerate}
	\item Naj bo $f: [a, b] \rightarrow \R$ \textit{zvezna} in \textit{nenegativna}. Ploščina območja, ki ga določa graf funkcije $y = f(x)$ nad intervalom $[a, b]$ na osi $x$, je
	$$\int_a^b f(x)\,dx ~=~ \int_a^b y\,dx.$$
	
	\item Naj bo $g: [c, d] \rightarrow \R$ \textit{zvezna} in \textit{nenegativna}. Ploščina območja, ki ga določa graf funkcije $x = g(y)$ na osi $y$, je
	$$\int_c^d g(y)\,dy ~=~ \int_c^d x\,dy.$$
\end{enumerate}
\end{trditev}
\vspace{0.5cm}

\begin{trditev}

Naj bo $F: [a, b] \rightarrow \R^2$ \textit{zvezno odvedljiva} pot, $F(t) = (x(t), y(t))$.
\begin{enumerate}
	\item Če je $y(t) \geq 0$ $\forall t$ in je $x(a) = \min{x(t)}$ ter $x(b) = \max{x(t)},$ potem je ploščina med krivuljo in osjo $x$ nad intervalom $[x(a), x(b)]$ enaka
	$$\int_a^b y(t)\,\dot{x}(t)\,dt.$$
	
	\item Če je $x(t) \geq 0 $ $\forall t$ in je $y(a) = \min{y(t)}$ ter $y(b) = \max{y(t)}$, potem je ploščina med krivuljo in osjo $y$ nad intervalom $[y(a), y(b)]$ enaka
	$$\int_a^b x(t)\,\dot{y}(t)\,dt.$$
\end{enumerate}
\end{trditev}
\vspace{0.5cm}

\begin{definicija}

Naj bo $F:[a, b] \rightarrow \mathbb{R}^2$ regularna parametrizacija krivulje K. Potem $F$ določa \textit{usmerjenost} K, določeno s smerjo, v kateri potuje točka $F(t)$ po $K$, ko potuje $t$ od $a$ do $b$. \\

\noindent \textit{Gladka enostavna sklenjena krivulja} je krivulja $K$, ki ima regularno \\parametrizacijo $F:[a, b] \rightarrow \mathbb{R}^2$, za katero velja $F(a) = F(b)$ in $\dot{F}(a) = \dot{F}(b)$, $F |_{[a, b)}$ pa je \textit{injektivna}. \\

\noindent Naj bo $A$ območje, ki ga omejuje \textit{gladka enostavna sklenjena krivulja} $K$. \\Regularna parametrizacija $F$ krivulje $K$ določa \textit{pozitivno usmerjenost} \hbox{krivulje} $K$, če je $A$ na levi strani, ko se pomikamo vzdolž $K$ v smeri usmerjenosti, ki jo določa $F$.

\end{definicija}
\vspace{0.5cm}

\begin{trditev}

Naj bo $F: [a, b] \rightarrow \R^2$, $F(t) = (x(t), y(t))$ \textit{regularna} parametrizacija \textit{enostavne sklenjene} krivulje $K$, ki določa pozitivno usmerjenost $K$. Potem je ploščina območja $A$ znotraj $K$ enaka
$$\int_a^b x(t)\,\dot{y}(t)\,dt ~=~ -\int_a^b y(t)\,\dot{x}(t)\,dt ~=~ \frac{1}{2} \int_a^b (x(t)\,\dot{y}(t) - y(t)\,\dot{x}(t))\,dt.$$

\end{trditev}
\vspace{0.5cm}

\begin{opomba}

Zgornja trditev velja tudi za \textit{odsekoma zvezno odvedljive} poti $F$, torej za zvezne $F(t) = (x(t), y(t))$, kjer sta odvoda $\dot{x}(t)$ in $\dot{y}(t)$ \textit{odsekoma zvezni} funkciji.

\end{opomba}
\vspace{0.5cm}

\begin{trditev}

Naj bo $r = r(\varphi)$ za $\varphi \in [\alpha, \beta]$ \textit{zvezno polarno} podana \hbox{krivulja}. Potem je ploščina območja, ki ga določa krivulja skupaj z \hbox{daljicama} 
$$\varphi = \alpha, ~~0 \leq r \leq r(\alpha) ~~~\text{in}~~~ \varphi = \beta, ~~0 \leq r \leq r(\beta),$$
enaka
$$\frac{1}{2} \int_\alpha^\beta r(\varphi)^2\,d\varphi.$$

\end{trditev}
\vspace{0.5cm}

% *************************************************************************************************

\subsection{Krivinska krožnica in ukrivljenost krivulje}
\vspace{0.5cm}

\begin{definicija}

Naj bo $K \subset \R^2$ krivulja, ki ima tangento v točki $(a, b) \in K$. \textit{Krivinska krožnica} na $K$ v točki $(a, b)$ je limitna lega krožnic, ki so tangentne na $K$ v $(a, b)$ ter gredo skozi točko $(x, y) \in K$, ko pošljemo $(x, y)$ proti $(a, b)$, če ta limita obstaja. Polmer krivinske krožnice imenujemo \textit{krivinski polmer}, njegovo obratno vrednost pa imenujemo \textit{ukrivljenost} krivulje v točki $(a, b)$.

\end{definicija}
\vspace{0.5cm}

\begin{trditev}
~
\begin{enumerate}

	\item[(1)] Naj bo $f$ \textit{dvakrat zvezno odvedljiva} funkcija. \textit{Ukrivljenost} grafa $f$ v točki $x$ je
	$$\kappa(x) ~=~ \frac{f''(x)}{(1+(f'(x))^2)^\frac{3}{2}}.$$
	
	\item[(2)] Naj bo $F$ \textit{dvakrat zvezno odvedljiva} pot. \textit{Ukrivuljenost tira} poti $F$ v točki $t$ je
	$$\kappa(t) ~=~ \frac{\ddot{y}(t) \dot{x}(t) - \dot{y}(t) \ddot{x}(t)}{(\dot{x}(t)^2 + \dot{y}(t)^2)^\frac{3}{2}}.$$
	
	\item[(3)] Naj  bo $r$ \textit{dvakrat zvezno odvedljiva} funkcija, ki določa polarno podano krivuljo. \textit{Ukrivljenost} krivulje v točki $\varphi$ je
	$$\kappa(\varphi) ~=~ \frac{r(\varphi)^2 + 2\dot{r}(\varphi)^2 - r(\varphi)\ddot{r}(\varphi)}{(r(\varphi)^2 + \dot{r}(\varphi)^2)^\frac{3}{2}}.$$

\end{enumerate}

\end{trditev}
\vspace{0.5cm}

% *************************************************************************************************

\subsection{Diferencialne enačbe v obliki diferenciala}
\vspace{0.5cm}

\begin{definicija}

\textit{Diferencialna enačba v obliki diferenciala} je enačba oblike 
$$P(x, y)\,dx ~+~ Q(x, y)\,dy ~=~ 0,$$
kjer sta $P, Q: A \rightarrow \mathbb{R}$ definirani na nekem območju $A \subset \mathbb{R}^2$. \\

\noindent Naj bosta $P, Q: A \rightarrow \mathbb{R}$ \textit{odvedljivi}. Diferencialna enačba \\$P(x, y)\,dx + Q(x, y)\,dy = 0$ je \textit{eksaktna} na $A$, če velja 
$$\frac{\partial P}{\partial y}(x, y) ~=~ \frac{\partial Q}{\partial x}(x, y), ~~~\forall x \in A.$$

\end{definicija}
\vspace{0.5cm}

\begin{trditev}

Naj bo $A \subset \R^2$ \textit{pravokotnik} in $P, Q: A \rightarrow \R$ \textit{zvezno odvedljivi} funkciji. Če je enačba $P(x, y)\,dx + Q(x, y)\,dy = 0$ \textit{eksaktna} na $A$, potem ima splošno rešitev oblike 
$$g(x, y) ~=~ C,$$
kjer je $g: A \rightarrow \R$ \textit{dvakrat zvezno odvedljiva} funkcija, ki zadošča
$$\frac{\partial g}{\partial x}(x, y) ~=~ P(x, y) ~~~\text{in}~~~ \frac{\partial g}{\partial y}(x, y) ~=~ Q(x, y).$$

\end{trditev}
\vspace{0.5cm}

\begin{definicija}[Integral s parametrom]

Naj bo $A = [a, b] \times [c, d] \subset \mathbb{R}^2$ pravokotnik in naj bo $f: A \rightarrow \mathbb{R}$ \textit{zvezna} funkcija. Funkcijo $F:[a, b] \rightarrow \mathbb{R}$, definirano s predpisom 
$$F(x) ~=~ \int_{c}^{d} f(x, y)\,dy,$$
imenujemo \textit{integral s parametrom}.

\end{definicija}
\vspace{0.5cm}

\begin{lema}

Pri oznakah kot v definiciji naj bo $f$ \textit{zvezno odvedljiva} funkcija dveh spremenljivk. Tedaj je integral s parametrom $x \mapsto F(x)$ \textit{zvezno odvedljiva} funkcija na $[a, b]$ in velja
$$F'(x) ~=~ \int_c^d \frac{\partial f(x, y)}{\partial x}\,dy.$$

\end{lema}
\vspace{0.5cm}

\begin{definicija}[Integrirajoči množitelj]

Naj bo dana diferencialna enačba $P(x, y)~dx + Q(x, y)~dy = 0$, kjer sta $P, Q: A \rightarrow \mathbb{R}$ \textit{zvezno odvedljivi} funkciji. Če je $\mu: A \rightarrow \mathbb{R}$ takšna \textit{zvezno odvedljiva} funkcija, da je enačba
$$\mu(x, y)\,P(x, y)\,dx ~+~ \mu(x, y)\,Q(x, y)\,dy ~=~ 0$$
\textit{eksaktna}, potem funkcijo $\mu$ imenujemo \textit{integrirajoči množitelj} dane enačbe.

\end{definicija}
\vspace{0.5cm}

\begin{trditev}

Če je $\mu$ \textit{integrirajoči množitelj} diferencialne enačbe \\$P(x, y)\,dx + Q(x, y)\,dy ~=~ 0$, potem zadošča 
$$\mu_y P - \mu_x Q ~=~ \mu(Q_x - P_y).$$
Dana enačba ima integrirajoči množitelj, ki je odvisen le od spremenljivke $x$, če ke izraz $\cfrac{Q_x - P_y}{Q}$ odvisen le od spremenljivke $x$ (ne pa od od $y$); v tem primeru je $\mu = \mu(x)$ rešitev enačbe
$$\frac{1}{\mu} \frac{d\mu}{dx} ~=~ \frac{P_y - Q_x}{Q}.$$
Dana enačba ima integrirajoči množitelj, ki je odvisen le od spremenljivke $y$, če ke izraz $\cfrac{Q_x - P_y}{P}$ odvisen le od spremenljivke $y$ (ne pa od od $x$); v tem primeru je $\mu = \mu(y)$ rešitev enačbe
$$\frac{1}{\mu} \frac{d\mu}{dy} ~=~ \frac{Q_x - P_y}{P}.$$


\end{trditev}
\vspace{0.5cm}

% *************************************************************************************************

\pagebreak

% #################################################################################################

\section{ŠTEVILSKE VRSTE}
\vspace{0.5cm}

% *************************************************************************************************

\subsection{Osnovni pojmi}
\vspace{0.5cm}

\begin{definicija}[Številska vrsta]

\textit{Številska vrsta} je vsote (neskončnega) \hbox{zaporedja} realnih števil. Če je $(a_n)_{n \in \mathbb{N}}$ zaporedje realnih števil, je \hbox{pripadajoča} številska vrsta
$$\sum_{n=1}^{\infty} a_n = a_1 + a_2 + \ldots + a_n + \ldots .$$
Za naravno število $k$ je \textit{$k$-ta delna vsota} vrste $\mathlarger{\sum_{n=1}^{\infty} a_n}$ enak
$$s_k := \sum_{n=1}^{k} a_n = a_1 + a_2 + \ldots + a_k.$$
Vrsta $\mathlarger{\sum_{n=1}^{\infty} a_n}$ je \textit{konvergentna (divergentna)}, če je konvergentno (divergentno) zaporedje delnih vsot $(s_k)_{k \in \mathbb{N}}$. Če je vrsta konvergentna, za njeno vsoto vzamemo limito delnih vsot.

\end{definicija}
\vspace{0.5cm}

\begin{lema}[Potrebni pogoj za kovergenco vrste]

Če vrsta $\mathlarger{\sum_{n=1}^\infty a_n}$ \textit{konvergira}, potem je 
$$\lim_{n \rightarrow \infty} a_n ~=~ 0.$$

\end{lema}
\vspace{0.5cm}

\begin{opomba}

Prvih nekaj členov vrste ne vpliva na konvergenco.

\end{opomba}
\vspace{0.5cm}

\pagebreak

% *************************************************************************************************

\subsection{Vrste s pozitivnimi členi}
\vspace{0.5cm}

\begin{trditev}[Primerjalni kriterij]

Naj za zaporedji števil $(a_n)_n$ in $(b_n)_n$ velja $0 \leq a_n \leq b_n$ $\forall n \in \N$.
\begin{enumerate}

	\item[(1)] Če $\mathlarger{\sum_{n=1}^\infty b_n}$ \textit{konvergira}, potem tudi $\mathlarger{\sum_{n=1}^\infty a_n}$ \textit{konvergira}.
	
	\item[(2)] Če $\mathlarger{\sum_{n=1}^\infty a_n}$ \textit{divergira}, potem tudi $\mathlarger{\sum_{n=1}^\infty b_n}$ \textit{divergira}.

\end{enumerate}
Rečemo, da je vrsta $\mathlarger{\sum_{n=1}^\infty b_n}$ \textit{majoranta} za vrsto $\mathlarger{\sum_{n=1}^\infty a_n}$, ter da je vrsta $\mathlarger{\sum_{n=1}^\infty a_n}$ \textit{minoranta} za vrsto $\mathlarger{\sum_{n=1}^\infty b_n}$.

\end{trditev}
\vspace{0.5cm}

\begin{trditev}[Kvocientni ali d'Alembertov kriterij]

Naj bo $(a_n)_n$ zaporedje \textit{pozitivnih} števil, za katerega obstaja
$$d ~:=~ \lim_{n \rightarrow \infty} \frac{a_{n+1}}{a_n}.$$
Potem vrsta $\sum_{n=1}^\infty a_n ~ \begin{cases}
\textit{konvergira}\,; ~&d<1\\
\textit{divergira}\,; ~&d>1 \\
\textit{konvergira ali divergira}\,; ~&d=1
\end{cases}.$ 

\end{trditev}
\vspace{0.5cm}

\begin{trditev}[Korenski ali Cuchyjev kriterij]

Naj bo $(a_n)_n$ zaporedje \\\hbox{\textit{nenegativnih}} števil, za katerega obstaja
$$c ~:=~ \lim_{n \rightarrow \infty} \sqrt[n]{a_n}.$$
Potem vrsta $\sum_{n=1}^\infty a_n ~ \begin{cases}
\textit{konvergira}\,; ~&c<1\\
\textit{divergira}\,; ~&c>1 \\
\textit{konvergira ali divergira}\,; ~&c=1
\end{cases}.$ 

\end{trditev}
\vspace{0.5cm}

\begin{trditev}[Integralski kriterij]

Naj bo $f: [1, \infty) \rightarrow \R$ \textit{zvezna, pozitivna} in \textit{padajoča} funkcija. Potem vrsta $\mathlarger{\sum_{n=1}^\infty f(n)}$ \textit{konvergira} natanko tedaj, ko \textit{konvergira} posplošeni integral
$$\int_1^\infty f(x)\,dx.$$

\end{trditev}
\vspace{0.5cm}

\begin{posledica}

Naj bo $p>0$ \textit{realno} število. Vrsta $\mathlarger{\sum_{n=1}^\infty \frac{1}{n^p}} ~ \begin{cases}
\textit{konvergira}\,; ~&p>1 \\
\textit{divergira}\,; ~&p \leq 1
\end{cases}.$

\end{posledica}
\vspace{0.5cm}

\begin{trditev}[Raabejev kriterij]

Naj bo $(a_n)_n$ zaporedje \textit{pozitivnih} števil, za katerega obstaja
$$r ~:=~ \lim_{n \rightarrow \infty} n \left( \frac{a_n}{a_{n+1}} - 1 \right).$$
Potem vrsta $\sum_{n=1}^\infty a_n ~ \begin{cases}
\textit{konvergira}\,; ~&r>1 \\
\textit{divergira}\,; ~&r<1 \\
\textit{konvergira ali divergira}\,; ~&r=1
\end{cases}.$

\end{trditev}
\vspace{0.5cm}

% *************************************************************************************************

\subsection{Alternirajoče vrste}
\vspace{0.5cm}

\begin{definicija}[Alternirajoča vrsta]

Vrsta $\mathlarger{\sum_{n=1}^{\infty} a_n}$ je \textit{alternirajoča}, če je člen $a_{n+1}$ nasprotno predznačen kot člen $a_n$ $\forall n \in \mathbb{N}$.

\end{definicija}
\vspace{0.5cm}

\begin{trditev}[Leibnizov kriterij]

Naj bo $\mathlarger{\sum_{i=1}^n a_n}$ \textit{alternirajoča} vrsta. Če \hbox{zaporedje} $(|a_n|)_n$ \textit{monotono} pada proti $0$, potem je vrsta $\mathlarger{\sum_{i=1}^n a_n}$ \hbox{\textit{konvergentna}}. Če v tem primeru $s_k$ označuje $k$-to delno vsoto vrste in $s$ vsoto vrste, potem velja
$$|s - s_k| ~\leq~ |a_{k+1}|.$$

\end{trditev}
\vspace{0.5cm}

% *************************************************************************************************

\subsection{Absolutna konvergenca}

\begin{definicija}

Vrsta $\mathlarger{\sum_{n=1}^{\infty} a_n}$ je \textit{absolutno konvergentna}, če je \mbox{\textit{konvergentna}} vrsta $\mathlarger{\sum_{n=1}^{\infty} |a_n|}$ iz absolutnih vrednosti členov vrste. Če je vrsta $\mathlarger{\sum_{n=1}^{\infty} a_n}$ \mbox{\textit{konvergentna}}, ni pa absolutno konvergenta, rečemo, da je \mbox{\textit{pogojno konvergentna}}.

\end{definicija}
\vspace{0.5cm}

\begin{trditev}

Če je vrsta \textit{absolutno konvergentna}, potem je tudi \mbox{\textit{konvergentna}}.

\end{trditev}
\vspace{0.5cm}

\begin{trditev}[Konvergentne vrste tvorijo vektorski prostor]

Naj bosta \hbox{vrsti} $\mathlarger{\sum_{n=1}^\infty a_n}$ in $\mathlarger{\sum_{n=1}^\infty b_n}$ \textit{(absolutno) konvergentni} in naj bo $c \in \R$. Potem so \hbox{\textit{(absolutno)} konvergentne} tudi vrste $\mathlarger{\sum_{n=1}^\infty (a_n \pm b_n)}$ in $\mathlarger{\sum_{n=1}^\infty c \cdot a_n}$ ter velja
$$\sum_{n=1}^\infty (a_n \pm b_n) ~=~ \sum_{n=1}^\infty a_n + \sum_{n=1}^\infty b_n ~~~\text{in}~~~ \sum_{n=1}^\infty c \cdot a_n ~=~ c \sum_{n=1}^\infty a_n.$$

\end{trditev}
\vspace{0.5cm}

\begin{trditev}

Naj bo $\mathlarger{\sum_{n=1}^\infty a_n}$ \textit{absolutno konvergentna} in $(b_n)_n$ \textit{omejeno} \hbox{zaporedje} števil.  Potem je $\mathlarger{\sum_{n=1}^\infty a_n b_n}$ tudi \textit{absolutno konvergentna}.\footnote{Analogna trditev za \textit{konvergentne} vrste ne velja.}

\end{trditev}
\vspace{0.5cm}

\begin{izrek}[Zamenjava vrstnega reda členov]
~
\begin{enumerate}

\item[(1)] Naj bo $\mathlarger{\sum_{n=1}^\infty a_n}$ \textit{absolutno konvergentna} vrsta. Potem je za poljubno \hbox{\textit{bijektivno}} funkcijo $\sigma: \N \rightarrow \N$ tudi vrsta $\mathlarger{\sum_{n=1}^\infty a_{\sigma(n)}}$ \hbox{\textit{absolutno konvergentna}} in ima isto vsoto kot začetna vrsta.

\item[(2)] Naj bo $\mathlarger{\sum_{n=1}^\infty a_n}$ \textit{pogojno konvergentna}. Potem $\forall s \in [-\infty, \infty]$ obstaja \textit{bijektivna} funkcija $\sigma: \N \rightarrow \N$, za katero je $\mathlarger{\sum_{n=1}^\infty a_{\sigma(n)}} = s$.

\end{enumerate}

\end{izrek}
\vspace{0.5cm}

% *************************************************************************************************

\pagebreak

% #################################################################################################

\section{FUNKCIJSKA ZAPOREDJA IN FUNKCIJSKE VRSTE}
\vspace{0.5cm}

% *************************************************************************************************

\subsection{Konvergenca funkcijskih zaporedij}
\vspace{0.5cm}

\begin{definicija}[Funkcijsko zaporedje]

Naj bo $A \subset \mathbb{R}$ in naj bo $\forall n \in \mathbb{N}$ dana funkcija $f_n: A \rightarrow \mathbb{R}$. Tedaj funkcije $f_n$ sestavljajo \textit{funkcijsko zaporedje} $(f_n)_{n \in \mathbb{N}}$. \\

Če $\forall a \in A$ obstaja limita številskega zaporedja $(f_n(a))_n$, rečemo, da funkcijsko zaporedje \textit{konvergira po točkah} na $A$. V tem primeru za $a \in A$ označimo 
$$f(a) ~:=~ \lim_{n \rightarrow \infty} f_n(a).$$
Tako dobljeno funkcijo $f: A \rightarrow \mathbb{R}$ imenujemo \textit{limitna funkcija} zaporedja $(f_n)_n$; pišemo 
$$f ~=~ \lim_{n \rightarrow \infty} f_n$$
in rečemo, da zaporedje $(f_n)_n$ \textit{konvergira k $f$ po točkah}.

\end{definicija}
\vspace{0.5cm}

\begin{opomba}

Funkcijsko zaporedje $(f_n: A \rightarrow \mathbb{R})_n$ v splošnem konvergira le v točkah iz neke podmnožice $B$ definicijskega območja $A$; množico $B$ \hbox{imenujemo} \textit{konvergenčno območje} funkcijskega zaporedja.

\end{opomba}
\vspace{0.5cm}

\begin{definicija}[Enakomerna konvergenca funkcijskega zaporedja]

Naj \hbox{funkcijsko} zaporedje $(f_n: A \rightarrow \mathbb{R})_n$ konvergira po točkah proti limitni funkciji \hbox{$f: A \rightarrow \mathbb{R}$}, torej $\forall x \in A$ in $\forall \varepsilon > 0$ obstaja tak $n_{x,\varepsilon}$, da $\forall n \geq n_{x,\varepsilon}$ velja 
$$|f_n(x) - f(x)| ~<~ \varepsilon.$$
Pravimo, da zaporedje $(f_n)_n$ \textit{konvergira proti $f$ enakomerno na $A$}, če $\forall  \varepsilon > 0$ obstaja tak $n_{\varepsilon}$, da $\forall n \geq n_{\varepsilon}$ in $\forall x \in A$ velja
$$|f_n(x) - f(x)| ~<~ \varepsilon.$$

\end{definicija}
\vspace{0.5cm}

\begin{izrek}[Zveznost limitne funkcije]

Naj bo $(f_n: A \rightarrow \R)_n$ zaporedje \textit{zveznih} funkcij, ki konvergira proti $f: A \rightarrow \R$ \textit{enakomerno} na $A$. Potem je $f$ \textit{zvezna}.

\end{izrek}
\vspace{0.5cm}

\begin{definicija}[Funkcijska vrsta]

Naj bo dano zaporedje funkcij \hbox{$(f_n: A \rightarrow \mathbb{R})_n$}. Vsoto $\mathlarger{\sum_{n=1}^{\infty} f_n}$ imenujemo \textit{funkcijska vrsta}. \\

Funkcijska vrsta $\mathlarger{\sum_{n=1}^{\infty} f_n}$ \textit{konvergira po točkah na $A$}, če zaporedje delnih vsot $s_k = \mathlarger{\sum_{n=1}^{k} f_n}$ konvergira po točah na A; to pomeni, da $\forall a \in A$ konvergira številska vrsta $\mathlarger{\sum_{n=1}^{\infty} f_n(a)}.$ \\

Naj bo $s: A \rightarrow \mathbb{R}$ limitna funkcija zaporedja delnih vsot $(s_k)_k$. \hbox{Funkcijska} vrsta $\mathlarger{\sum_{n=1}^{\infty}f_n}$ \textit{konvergira k $s: A \rightarrow \mathbb{R}$ enakomerno na A}, če zaporedje delnih vsot $(s_k)_k$ konvergira k $s$ enakomerno na $A$.
 

\end{definicija}
\vspace{0.5cm}

\begin{posledica}

Če so funkcije $(f_n: A \rightarrow \mathbb{R})_n$ zvezne in funkcijska vrsta $\mathlarger{\sum_{n=1}^{\infty} f_n}$ konvergira k $s: A \rightarrow \mathbb{R}$ enakomerno na $A$, potem je $s$ zvezna na $A$.

\end{posledica}
\vspace{0.5cm}

% *************************************************************************************************

\subsection{Enakomerno konvergentne vrste}
\vspace{0.5cm}

\begin{trditev}[Weierstrassov kriterij]

Naj bo $(f_n: A \rightarrow \R)_n$ zaporedje funkcij in naj obstaja tako zaporedje \textit{pozitivnih} števil $(c_n)_n$, da $\forall n \in \N$ velja
$$|f_n(x)| ~\leq~ c_n ~~~\forall x \in A.$$
Če je številska vrsta $\mathlarger{\sum_{n=1}^\infty c_n}$ \textit{konvergentna}, potem funkcija vrsta $\mathlarger{\sum_{n=1}^\infty f_n}$ \textit{\\\hbox{konvergira} enakomerno} in \textit{absolutno} na $A$. Če so funkcije $f_n$ \textit{zvezne}, je tudi vsota vrste \textit{zvezna} funkcija.

\end{trditev}
\vspace{0.5cm}

\begin{izrek}[Integriranje po členih]

Naj bo $J \subset \R$ \textit{omejen} interval, \hbox{$a \in J$}, in naj bo $(f_n: J \rightarrow \R)_n$ zaporedje \textit{zveznih} funkcij na $J$. Če vrsta $\mathlarger{\sum_{n=1}^\infty f_n}$ \textit{\hbox{konvergira} enakomerno} na $J$, potem tudi $\mathlarger{\sum_{n=1}^\infty \int_a^x f_n(t)\,dt}$ \textit{\hbox{konvergira} \hbox{enakomerno}} za $x \in J$ in $\forall x \in J$ velja
$$\int_a^x \left( \sum_{n=1}^\infty f_n(t) \right)\,dt ~=~ \sum_{n=1}^\infty \int_a^x f_n(t)\,dt.$$

\end{izrek}
\vspace{0.5cm}

\begin{izrek}[Odvajanje po členih]

Naj bo $J \subset \R$ interval in naj bo \hbox{$(f_n: J \rightarrow \R)_n$} zaporedje \textit{zvezno odvedljivih} funkcij na $J$. Če vrsta $\mathlarger{\sum_{n=1}^n f_n}$ \textit{konvergira po točkah} na $J$ in vrsta iz odvodov $\mathlarger{\sum_{n=1}^\infty f'_n}$ \textit{konvergira enakomerno} na $J$, potem je $\mathlarger{\sum_{n=1}^\infty f_n}$ \textit{odvedljiva} na $J$ in $\forall x \in J$ velja
$$\frac{d}{dx} \left( \sum_{n=1}^\infty f_n(x) \right) ~=~ \sum_{n=1}^\infty f'_n(x).$$

\end{izrek}
\vspace{0.5cm}

% *************************************************************************************************

\subsection{Potenčne vrste}
\vspace{0.5cm}

\begin{definicija}[Potenčna vrsta]

Naj bo $(a_n)_{n \geq 0}$ \textit{realno zaporedje} in $c \in \mathbb{R}$. Funkcijsko zaporedje $$\sum_{n=0}^{\infty} a_n (x - c)^n$$
imenujemo \textit{potenčna vrsta s središčem v $c$}.

\end{definicija}
\vspace{0.5cm}

\begin{opomba}
~
\begin{enumerate}
	\item Potenčna vrsta vedno konvergira v središču $c$.
	\item Središče $c$ lahko vedno prestavimo v $0$ z uvedbo nove spremenljivke $t = x - c$:
	$$\sum_{n=0}^{\infty} a_n (x-c)^n ~=~ \sum_{n=0}^{\infty} a_n t^n.$$
\end{enumerate}
\end{opomba}
\vspace{0.5cm}

\begin{trditev}

Naj bo dana potenčna vrsta $\mathlarger{\sum_{n=0}^\infty a_n x^n}$ in naj bo $b > 0$.  
\begin{enumerate}
	\item[(1)] Če vrsta \textit{absolutno konvergira} pri $x = b$ ali $x = -b$, potem \textit{konvergira absolutno} in \textit{enakomerno} za $x \in [-b, b]$.
	\item[(2)] Če vrsta \textit{konvergira} pri $x = b$ ali $x = -b$, potem \textit{absolutno konvergira} $\forall x \in (-b, b)$.
	\item[(3)] Če vrsta \textit{divergira} pri $x = b$ ali $x = -b$, potem \textit{divergira} \\$\forall x \in (-\infty, -b) \cup (b, \infty)$.
\end{enumerate}

\end{trditev}
\vspace{0.5cm}

\begin{definicija}[Konvergenčni polmer]

Naj bo dana potenčna vrsta $\mathlarger{\sum_{n=0}^{\infty} a_n x^n}$. Število
$$R ~:=~ \sup \{b \geq 0 \mid \text{vrsta konvergira pri } b \} ~\in~ [0, \infty)$$
imenujemo \textit{konvergenčni polmer} potenčne vrste. Konvergenčno območje potenčne vrste označimo z $D$.

\end{definicija}
\vspace{0.5cm}

\begin{posledica}

Naj bo $R > 0$ \textit{konvergenčni polmer potenčne vrste} $\mathlarger{\sum_{n=0}^{\infty} a_n x^n}$. Potem vrsta \textit{absolutno konvergira} za $|x| < R$. Vrsta enakomerno konvergira na vsakem manjšem intervalu $|x| \leq r < R$. Vsota potenčne vrste je zvezna funkcija na $(-R, R)$. Vrsta divergira $\forall x$, $|x| > R$; v krajiščih intervala lahko potenčna vrsta konvergira ali divergira. Torej je
$$(-R, R) ~\subseteq~ D ~\subseteq~ [-R, R].$$

\end{posledica}
\vspace{0.5cm}

\begin{trditev}[Konvergenčni polmer]

Naj bo dana \textit{potenčna vrsta} $\mathlarger{\sum_{n=0}^\infty a_n x^n}$. Za konvergenčni polmer velja:
\begin{enumerate}
	\item[(1)] $\mathlarger{\frac{1}{R} = \lim_{n \rightarrow \infty} \frac{|a_{n+1}|}{|a_n|}}$, če ta limita obstaja;
	\item[(2)] $\mathlarger{\frac{1}{R} = \lim_{n \rightarrow \infty} \sqrt[n]{|a_n|}}$, če ta limita obstaja.
\end{enumerate}

\end{trditev}
\vspace{0.5cm}

\begin{definicija}[Limes superior]

Naj bo $(b_n)_n$ \textit{zaporedje realnih števil}. Največje stekališče zaporedja $(b_n)_n$ označimo 
$$\lim_{n \rightarrow \infty} \sup b_n ~\in~ [-\infty, \infty]$$
in ga imenujemo \textit{limes superior}.

\end{definicija}
\vspace{0.5cm}

\begin{izrek}[Cauchy-Hadamard]

Za konvergenčni polmer $R$ potenčne vrste $\sum_{n=0}^\infty a_n x^n$ velja
$$\frac{1}{R} ~=~ \lim_{n \rightarrow \infty} \sup{\sqrt[n]{|a_n|}}.$$

\end{izrek}
\vspace{0.5cm}

\begin{izrek}[Abelov izrek]

Naj bo $R$ konvergenčni polmer potenčne vrste $\mathlarger{\sum_{n=0}^\infty a_n x^n}$. Če vrsta \textit{konvergira} za $x = R$ ($x = -R$), potem je vsota vrste \textit{zvezna} v $R$ ($-R$).

\end{izrek}
\vspace{0.5cm}

\begin{izrek}[Odvajanje in integriranje potenčnih vrst]

Naj bo $R > 0$ \\\hbox{konvergenčni} polmer potenčne vrste $\mathlarger{f(x) = \sum_{n=0}^\infty a_n x^n}$. Potem imata vrsti, ki ju dobimo s členim odvajanjem $\mathlarger{\sum_{n=1}^\infty n a_n x^{n-1}}$ in členim integriranjem $\sum_{n=0}^\infty \frac{a_n}{n+1} x^{n+1}$ tudi konvergenči polmer $R$ in $\forall x \in (-R, R)$ velja:
$$f'(x) ~=~ \sum_{n=1}^\infty n a_n x^{n-1} ~~~\text{in}~~~ \int_0^x f(t)\,dt ~=~ \sum_{n=0}^\infty \frac{a_n}{n+1} x^{n+1}.$$

\end{izrek}
\vspace{0.5cm}

\begin{posledica}

Vsota potenčne vrste je poljubno mnogokrat \textit{zvezno odvedljiva} na $(-R, R)$, kjer je $R$ konvergenčni polmer vrste.

\end{posledica}
\vspace{0.5cm}

% *************************************************************************************************

\subsection{Taylorjeva vrsta in analitične funkcije}
\vspace{0.5cm}

\begin{definicija}[Taylorjeva vrsta]

Naj bo $f$ \textit{poljubno mnogokrat zvezno odvedljiva} v okolici točke $c \in \mathbb{R}$. \textit{Taylorjeva vrsta funkcije $f$ s središčem v $c$} je 
$$T_c(x) ~=~ \sum_{n=0}^{\infty} \frac{f^{(n)}(c)}{n!} (x-c)^n.$$ \\

Naj bo $J \subset \mathbb{R}$ \textit{odprt interval} in $f \in \mathcal{C}^{\infty}(J)$. Rečemo, da je $f$ \textit{analitična} na $J$, če je $\forall c \in J$ Taylorjeva vrsta $T_c$ enaka funkciji $f$ na neki okolici \hbox{točke $c$}.

\end{definicija}
\vspace{0.5cm}

\begin{definicija}[Analitična vrsta]

Če Taylorjeva vrsta funkcije $f$ \textit{konvergira} tudi v okolici točke $a$ proti funkciji $f$, rečemo, da je funkcija $f$ v okolici točke $a$ \textit{analitična}.

\end{definicija}
\vspace{0.5cm}

\begin{trditev}[Primeri Taylorjevih (McLaurinovih) razvojev]
~
\begin{enumerate}
	
	\item \textsc{Eksponentna funkcija}: 
	$$e^x ~=~ \sum_{n=0}^\infty \frac{x^n}{n!} ~=~ 1 + \frac{x}{1!} + \frac{x^2}{2!} + \frac{x^3}{3!} + \ldots ~~~\forall x \in \R$$
	
	\item \textsc{Sinus in kosinus}:
	\begin{align*}
	\sin{x} ~&=~ \sum_{n=0}^\infty \frac{(-1)^n}{(2n+1)!}x^{2n+1} ~=~ x - \frac{x^3}{x!} + \frac{x^5}{5!} - \frac{x^7}{7!} + \ldots ~~~\forall x \in \R \\
	\cos{x} ~&=~ \sum_{n=0}^\infty \frac{(-1)^n}{(2n!)}x^{2n} ~=~ 1 - \frac{x^2}{2!} + \frac{x^4}{4!} - \frac{x^6}{6!} + \ldots ~~~\forall x \in \R
	\end{align*}
	
	\item \textsc{Logaritemska funkcija}:
	$$\ln{1+x} ~=~ \sum_{n=1}^\infty \frac{(-1)^{n-1}}{n}x^n ~=~ x - \frac{x^2}{2} + \frac{x^3}{3} - \frac{x^4}{4} + \ldots ~~~\forall x \in (-1, 1]$$
	
	\item \textsc{Binomska vrsta}:
	\begin{align*}
	(1+x)^r ~=~ \sum_{n=0}^\infty \binom{r}{n}x^n ~=~ 1 + \binom{r}{1}x + \binom{r}{2}x^2 + \binom{r}{3}x^3 + \ldots ~~~ \forall x &\in (-1, 1) 
	\end{align*}	
	$r, n \in \N$.
	
\end{enumerate}

\end{trditev}
\vspace{0.5cm}

% *************************************************************************************************

\subsection{Fourierove vrste}
\vspace{0.5cm}

\begin{definicija}[Fourierova vrsta]

\textit{Fourierova vrsta} je funkcijska vrsta oblike
$$a_0 ~+~ \sum_{n=1}^{\infty} (a_n \cos(nx) + b_n \sin(nx)),$$
kjer sta $(a_n)_{n \geq 0}$ in $(b_n)_{n \geq 1}$ \textit{realni} zaporedji.

\end{definicija}
\vspace{0.5cm}

\begin{definicija}[Skalarni produkt]

Naj bosta $f, g: [a, b] \rightarrow \mathbb{R}$ \textit{odsekoma zvezni} funkciji. Izraz
$$\langle f, g \rangle ~:=~ \int_{a}^{b} f(x) g(x)\,dx$$
imenujemo \textit{skalarni produkt} funkcij $f$ in $g$. Če je $\langle f, g \rangle = 0$, rečemo, da sta $f$ in $g$ \textit{ortogonalni}. Izraz $\| f \| = \sqrt{\langle f, f \rangle}$ imenujemo \textit{norma} funkcije $f$.

\end{definicija}
\vspace{0.5cm}

\begin{trditev}

$\langle \cdot, \cdot \rangle$ ima lastnosti skalarnega produkta: za \textit{odsekoma zvezne} funkcija $f, g, h: [a, b] \rightarrow \R$ in skalarja $\lambda, \mu \in \R$ velja:
\begin{enumerate}
	
	\item[(1)] \textsc{Pozitivna definitnost}:	
	$$\langle f, f \rangle = 0 ~\iff~ f(x) = 0 ~~~\text{za vse razen morda končno mnogo}~ x$$
	
	\item[(2)] \textsc{Simetričnost}:
	$$\langle g, f \rangle ~=~ \langle f, g \rangle$$
	
	\item[(3)] \textsc{Bilinearnost}:
	$$\langle \lambda f + \mu g, h \rangle ~=~ \lambda \langle f, h \rangle ~+~ \mu \langle g, h \rangle$$	
	
\end{enumerate}

\end{trditev}
\vspace{0.5cm}

\begin{dogovor}

Za \textit{odsekoma zvezno} funkcijo $f$ v točkah nezveznosti vrednost funkcije v točki določimo kot
$$f(x) ~=~ \frac{1}{2} \left( \lim_{t \nearrow x} f(t) + \lim_{t \searrow x} f(t) \right).$$

\end{dogovor}
\vspace{0.5cm}

\begin{trditev}[Ortogonalnost sistema trigonometričnih funkcij]

Funkcije
$$\{ 1, \cos{x}, \sin{x}, \cos{(2x)}, \sin{(2x)}, \ldots \} ~=~ \{ 1, \cos{nx}, \sin{nx} \mid n \in \N \}$$
so \textit{paroma ortogonalne} na $[-\pi, \pi]$. Rečemo, da te funkcije sestavljajo \textit{\\\hbox{ortogonalen} sistem} na $[-\pi, \pi]$. Poleg tega velja še
$$\| 1 \|^2 ~=~ 2\pi, ~~~\| \cos{(nx)} \|^2 ~=~ \pi, ~~~\| \sin{(nx)} \|^2 ~=~ \pi ~~~\forall n \in \N.$$

\end{trditev}
\vspace{0.5cm}

\begin{lema}

Naj bosta zaporedji $(a_n)_{n \geq 0}$ in $(b_n)_{n \geq 1}$ takšni, da vrsta
$$a_0 ~+~ \sum_{n=1}^\infty (a_n \cos(nx) + b_n \sin(nx))$$
\textit{konvergira enakomerno} na $[-\pi, \pi]$. Označimo vsoto vrste s $f: [-\pi, \pi] \rightarrow \R$. Potem je
\begin{align*}
a_0 ~&=~ \frac{1}{2\pi} \int_{-\pi}^\pi f(x)\,dx ~=~ \frac{\langle f(x), 1 \rangle}{\| 1 \|^2}, \\
a_n ~&=~ \frac{1}{\pi} \int_{-\pi}^\pi f(x) \cos(nx)\,dx ~=~ \frac{\langle f(x), \cos(nx) \rangle}{\| \cos(nx) \|^2}, \\
b_n ~&=~ \frac{1}{\pi} \int_{-\pi}^\pi f(x) \sin(nx)\,dx ~=~ \frac{\langle f(x), \sin(nx) \rangle}{\| \sin(nx) \|^2} ~~~\forall n \in \N
\end{align*}

\end{lema}
\vspace{0.5cm}

\begin{opomba}
~
\begin{enumerate}

	\item[(1)] Izračunane formule za koeficiente $a_n$ in $b_n$ so smiselne za vsako \textit{\hbox{odsekoma} zvezno} funkcijo $f$. Lahko pa se zgodi, da tako dobljena vrsta \textit{divergira} ali pa, če \textit{konvergira}, da njena vsota ni enaka $f$.

	\item[(2)] Vsota Fourierove vrste je \textit{periodilna} funkcija s periodo $2\pi$. Zato je smiselno funkcijo $f: [\pi, \pi] \rightarrow \R$, ki jo razvijemo v Fourierovo vrsto, razširiti do periodične funkcije na $\R$.

\end{enumerate}

\end{opomba}
\vspace{0.5cm}

\begin{izrek}

Naj bo $f: \R \rightarrow \R$ \textit{dvakrat zvezno odvedljiva} funkcija, \hbox{periodična} s periodo $2\pi$. Potem Fourierova vrsta za $f$ \textit{konvergira enakomerno} in \textit{\hbox{absolutno}} na $\R$ in njena vsota je enaka $f$.

\end{izrek}
\vspace{0.5cm}

\begin{definicija}

Funkcija $f:[a, b] \rightarrow \mathbb{R}$ je \textit{odsekoma zvezno odvedljiva}, če je \textit{odsekoma zvezna} na $[a, b]$ in \text{zvezno odvedljiva} na $[a, b]$ razen morda v končno mnogo točkah, v katerih obstajata leva in desna limita odvoda.

\end{definicija}
\vspace{0.5cm}

\begin{opomba}

\textit{Odsekoma zvezno odvedljiva} funkcija ni odvedljiva v skokih in v točkah, kjer se graf $f$ zlomi, torej ima različni levo in desno tangento v tej točki. V takšni točki namreč res obstajata levi in desni odvod, saj na primer 
$$f'(x^-) ~=~ \lim_{t \nearrow x} \frac{f(t) - f(x)}{t-x} ~=~ \lim_{t \nearrow x} f'(t),$$
torej levi odvod obstaja, saj pbstaja leva limita odvodov; analogno velja za desni odvod.

Če torej v skoku spremenimo vrednost funkcije $f$ tako, da je enaka levi (desni) limiti $f$ v tej točki, potem obstaja levi (desni) odvod funckije $f$ v tej točki.

\end{opomba}
\vspace{0.5cm}

\begin{izrek}

Naj bo $f: [-\pi, \pi] \rightarrow \R$ \textit{odsekoma zvezno odvedljiva}. Potem Fourierova vrsta za $f$ \textit{konvergira} proti funkciji $f$ \textit{po točkah}; bolj natančno, v točki, kjer $f$ ni \textit{zvezna}, Fourierova vrsta \textit{konvergira} proti povprečni vrednosti leve in desne limite funkcije v tej točki.

\end{izrek}
\vspace{0.5cm}

\begin{lema}[Riemannova lema]

Naj bo $f: [a, b] \rightarrow \R$ \textit{odsekoma zvezna}. Potem je za $r \in \R$
$$\lim_{r \rightarrow \infty} \int_a^b f(x) \sin(rx)\,dx ~=~ 0.$$

\end{lema}
\vspace{0.5cm}

\begin{posledica}

Naj bo $f$ \textit{poljubno mnogokrat zvezno odvedljiva} funkcija na $\R$, periodična s periodo $2\pi$. Potem je $f$ enaka svoji Fourierovi vrsti, ki \textit{\hbox{konvergira} enakomerno} in \textit{absolutno} na $\R$. Vrsto lahko poljubno mnogokrat členoma \hbox{odvajamo} in vsota tako dobljene vrste je enaka utreznemu odvodu funkcije $f$.

\end{posledica}
\vspace{0.5cm}

\begin{definicija}

Ker lahko vsako \textit{odsekoma zvezno odvedljivo} funkcijo na $[-\pi, \pi]$ razvijemo v Fourierovo vrsto, ki konvergira proti $f$ (vsaj po točkah), rečemo, da funkcije 
$$ \{ 1, \cos(nx), \sin(nx) \mid n \in \mathbb{N} \}$$
sestavljajo \textit{poln sistem funkcij} na intervalu $[-\pi, \pi]$.

\end{definicija}
\vspace{0.5cm}

\begin{posledica}[Fourierova sinusna in kosinusna vrsta]

Naj bo $f:[0, \pi] \rightarrow \mathbb{R}$ \textit{odsekoma zvezno odvedljiva}. Potem lahko $f$ na intervalu $[0, \pi]$ razvijemo v \textit{sinusno} Fourierovo vrsto
$$f(x) ~=~ \sum_{n=1}^{\infty} b_n \sin(nx), ~~~b_n ~=~ \frac{2}{\pi} \int_{0}^{\pi} f(x) \sin(nx)\,dx$$
in v \textit{kosinusno} Fourierovo vrsto
\begin{align*}
f(x) ~=~ a_0 ~+~ \sum_{n=1}^{\infty} a_n \cos(nx), ~~~a_0 ~&=~ \frac{1}{\pi} \int_{0}^{\pi} f(x)\,dx, \\ a_n ~&=~ \frac{2}{\pi} \int_{0}^{\pi} f(x) \cos(nx)\,dx.
\end{align*}

\end{posledica}
\vspace{0.5cm}

\begin{posledica}[Kompleksna Fourierova vrsta]

Naj bo $f:[-\pi, \pi] \rightarrow \mathbb{R}$ \textit{\hbox{odsekoma} zvezno odvedljiva}. Potem lahko $f$ razvijemo v \textit{kompleksno} Fourierovo vrsto
$$f(x) ~=~ \sum_{n=-\infty}^{\infty} A_n e^{inx},$$
kjer je $i = \sqrt{-1}$ \textit{imaginarna enota}, koeficienti $A_n$ za $n \in \mathbb{Z}$ pa so dani z
$$A_n ~=~ \frac{1}{2\pi} \int_{-\pi}^{\pi} f(x) e^{-inx}\,dx.$$

\end{posledica}
\vspace{0.5cm}

\begin{posledica}[Razvoj v Fourierovo vrsto na drugih intervalih]

Naj bo \hbox{$f:[a, b] \rightarrow \mathbb{R}$} \textit{odsekoma zvezno odvedljiva}. Potem $f$ lahko razvijemo v Fourierovo vrsto po funkcijah
$$\{ 1, ~\cos \left( \frac{2n\pi}{b-a}x \right), ~\sin \left( \frac{2n\pi}{b-a}x \right) \mid n \in \mathbb{N} \}.$$
Tako dobimo
$$f(x) ~=~ a_0 ~+~ \sum_{n=1}^{\infty} \left( a_n \cos \left( \frac{2n\pi}{b-a}x \right) + b_n \sin \left( \frac{2n\pi}{b-a}x \right) \right),$$
kjer je 
\begin{align*}
a_0 ~&=~ \frac{1}{b-a} \int_{a}^{b} f(x)\,dx, \\
a_n ~&=~ \frac{2}{b-a} \int_{a}^{b} f(x) \cos \left( \frac{2n\pi}{b-a}x \right)\,dx, \\
b_n ~&=~ \frac{2}{b-a} \int_{a}^{b} f(x) \sin \left( \frac{2n\pi}{b-a}x \right)\,dx. \\
\end{align*}

\end{posledica}
\vspace{0.5cm}

% *************************************************************************************************

\pagebreak

% #################################################################################################

\section{NAVADNE DIFERENCIALNE ENAČBE}
\vspace{0.5cm}

% *************************************************************************************************

\subsection{Diferencialna enačba prvega reda}
\vspace{0.5cm}

\begin{definicija}

Radi bi poiskali \textit{enkrat odvedljivo} funkcijo, ki zadošča enačbi
$$y' = f(x, y).$$
Vsako tako funkcijo imenujemo \textit{rešitev} dane diferencialne enačbe, njen graf $y = y(x)$ pa \textit{rešitvena krivulja}.

Pravimo, da je z enačbo podano \textit{polje smeri}, v vsaki točki je predpisana smer, v kateri mora potekati rešitvena krivulja. To polje smeri grafično predstavimo kot družino krivulj \--- \textit{izoklin} \--- vzdolž katerih je smer \textit{konstantna}.

\end{definicija}
\vspace{0.5cm}

\begin{definicija}[Ortogonalne trajektorije]

\textit{Ortogonalne trajektorije} dane družine krivulj so take krivulje, ki v vsaki svoji točki sekajo tisto od krivulj dane družine, ki poteka skozi točko, pod pravim kotom.

Tej ortogonalni družini pripada diferencialna enačba, ki je v preprosti zvezi z diferencialno enačbo prvotne družine krivulj. V enačbi prvotne družine le zamenjamo $y'$ z $\cfrac{1}{y'}$ (kar določa pravokotno smer).

\end{definicija}
\vspace{0.5cm}

\begin{izrek}[Eksistenca in enoličnost]

Naj bo $D$ \textit{odprta} množica na ravnini, $(x_0, y_0) \in U$ poljubna točka in $f: D \rightarrow \R$ poljubna funkcija dveh spremenljivk, definirana na $D$.
\begin{enumerate}
	
	\item[(1)] Če je $f$ \textit{zvezna} funkcija, obstaja tak interval $I \subset \R$, in vsaj ena \textit{odvedljiva} funkcija $y = y(x)$, definirana na $I$, da je:
	\begin{enumerate}
		\item[(i)] $x_0 \in I$ in $y(x_0) = y_0$,
		\item[(ii)] $(x, y(x)) \in D$ $\forall x \in I$ in
		\item[(iii)] $y'(x) = f(x, y(x))$ $\forall x \in I$. 
	\end{enumerate}		
	
	\item[(2)] Če poleg tega obstaja v okolici točke $(x_0, y_0)$ tudi \textit{zvezen} parcialni odvod $\cfrac{\partial f}{\partial y}$, je funkcija $y = y(x)$ z zgornjimi lastnostmi ena sama.
	
\end{enumerate}

\end{izrek}
\vspace{0.5cm}

% *************************************************************************************************

\subsection{Linearna diferencialna enačba prvega reda}
\vspace{0.5cm}

\begin{izrek}

Splošnja rešitev \textit{linearne diferencialne enačbe prvega reda} je
$$y ~=~ e^{-\int_{x_0}^x f(t)\,dt} \left( \int_{x_0}^x g(s) e^{\int_{x_0}^x f(t)\,dt}\,ds + C \right).$$

\end{izrek}
\vspace{0.5cm}

\begin{opomba}

Včasih je koristno enačbo pomnožiti z ustreznim faktorjem, tako da na levi strani dobimo odvod neke funkcije.

\end{opomba}
\vspace{0.5cm}

% *************************************************************************************************

\subsection{Diferencialna enačba drugega reda}
\vspace{0.5cm}

% *************************************************************************************************

\pagebreak

% #################################################################################################

\end{document}